\chapter{Conclusiones}
\label{ch:concl}

En esta tesis se estudi\'o la medici\'on de neutrinos c\'osmicos ultra energ\'eticos mediante detectores de superficie. 
El trabajo fue dividido en dos partes que abordan tecnolog\'ias diferentes para estudiar las lluvias atmosf\'ericas que estos son capaces de producir.
La primera parte se trat\'o la b\'usqueda de neutrinos basada en la interacci\'on de las part\'iculas de la lluvia con el Detector de Superficie del Observatorio Pierre Auger.
En particular, se estudi\'o el procedimiento de identificaci\'on, las posibles fuentes de fondo, el m\'etodo desarrollado para calcular la exposici\'on y las incertezas sistem\'aticas asociadas en la b\'usqueda.
Asimismo, se present\'o por primera vez la integraci\'on de los distintos an\'alisis complementarios desarrollados en el experimento, que incluyen dos criterios e reconocimiento de neutrinos descendentes y uno de neutrinos rasantes, en un \'unico resultado.
Por otro lado tambi\'en se incluy\'o el efecto del envejecimiento del detector en el c\'alculo de la exposici\'on.
Ning\'un candiato a neutrino fue encontrado en los datos adquiridos entre el 1 de enero de 2004 y el 20 de junio de 2013. 
Suponiendo un flujo diferencial t\'ipico, $\Phi(E_\nu)= k\cdot E^{-2}_\nu$ y una relaci\'on entre sabores de $1 : 1 : 1$, se fij\'o un l\'imite sobre el flujo de neutrinos de cada sabor, de $k < 6.4 \times 10^{-9}~{\rm GeV~cm^{-2}~s^{-1}~sr^{-1}}$ con un nivel de confianza del 90\% en el rango de ${1.0 \times 10^{17}}$ {eV} - ${2.5 \times 10^{19}}$ {eV}, el m\'as estricto hasta la fecha.
Adem\'as, se observ\'o que la sensitividad m\'axima del detector del SD de Auger se alcanza a energ\'ia alrededor del EeV, lo que coincide con el m\'aximo flujo esperado proveniente de modelos cosmog\'enicos.
El l\'imite actual de Auger es un factor $\sim4$ m\'as chico que la cota superior de Waxman-Bachall para la producci\'on de neutrinos en fuentes \'opticamente finas (optically thin), siendo el primer detector de rayos c\'osmicos en alcanzar dicha cota.
Algunos modelos de producci\'on en fuentes astrof\'isicas, como AGN son excluidas con m\'as del \cant{90\%}{C.L.}.
Los modelos cosmog\'enicos que asumen protones inyectados en la fuente como primarios y modelos de evoluci\'on fuertes (tipo FRII) se encuentran fuertemente desfavorecidos en las observaciones de Auger, aunque no pueden ser descartados hasta el momento.
Sin embargo, los modelos que proponen n\'ucleos de hierro como primarios se encuentran muy por debajo del alcance del experimento, que necesitar\'ia ganar un factor 10 en exposici\'on para poder ser concluyente.
% Por \'ultimo, un gran n\'umero de modelos ex\'oticos son exluidos con \cant{99}{C.L.}.

En la segunda parte de este trabajo se estudi\'o el potencial de detecci\'on de las ondas de radio emitidas por lluvias atmosf\'ericas para medir las producidas por neutrinos.
Tambi\'en se evaluaron las capacidades y limitaciones de un detector de superficie conformado por antenas de radio a la hora de detectar flujos de neutrinos c\'osmicos.
Para ello se utilizaron simulaciones que incluyen la interacci\'on del neutrinos con la tierra, el decaimiento del lept\'on tau en la atm\'osfera, el subsiguiente desarrollo de la lluvia atmosf\'erica extendida con producci\'on de radiaci\'on electromagn\'etica en frecuencias de radio, y la generaci\'on de la se\~nal en las antenas del detector.
A partir de estas simulaciones se estudiaron algoritmos de reconstrucci\'on e identificaci\'on comprobando la factibilidad de la detecci\'on y logrando estimar las eficiencias.
Adem\'as se realiz\'o una caracterizaci\'on detallada de la huella de radio dejada por este tipo de eventos.
Tambi\'en se utilizaron los criterios definidos para calcular la exposici\'on en tres diferentes familias de topolog\'ias, regular, tipo panal de abeja y un arreglo con celdas de bordes densos, as\'i como para diferentes niveles de disparo local y global.
Con todo esto se estimaron las capacidades del detector en m\'as de 5000 combinaciones de sus par\'ametros y se compar\'o cada una con la alcanzada por Auger en una situaci\'on equivalente.
En algunas circunstancias el c\'alculo arroj\'o que el arreglo de 90000 antenas alcanza una exposici\'on dos \'ordenes de magnitud mayor que la de Auger.
Se encontr\'o que el nivel de disparo local es uno de los factores que m\'as impacta en la exposici\'on del detector.
De acuerdo a su valor el m\'aximo desempe\~no se alcanza a trav\'es de un compromiso entre conseguir una buena eficiencia a bajas energ\'ias (\cant{\sim10^{17.25}}{eV}), lo que se logra mediante detctores densos, y alcanzar una mayor superficie cubierta a trav\'es de separar las antenas.
Para el nivel de threshold local elegido en el c\'alculo del desempe\~no (\cant{50}{\mu V/m}) la p\'erdida de eficiencia a baja energ\'ia fue despreciable frente al aumento del \'area del detector. En estas condiciones se calcul\'o el l\'imite diferencial al \cant{90\%}{C.L.} para detectores de dos tama\~nos, \cant{L=250}{km} que corresponde al propuesto por GRAND y \cant{L=500}{km} que fue considerado como el m\'aximo valor viable.
Se calcul\'o la cantidad de eventos esperados para estos dos detectores se encontr\'o que incluso para \cant{L=250}{km} ser\'ia posible rechazar todos los modelos cosmog\'enicos que utilizan protones o combinaciones (modelos h\'ibridos) de primarios.
Adem\'as esste es el tama\~no m\'inimo necesario para comenzar a detectar modelos de Hierro a los que un detector de \cant{L=500}{km} podr\'ia rechazar con significancias considerables.
Por \'ultimo tambi\'en se estudiaron los l\'imites de la t\'ecnica, calculando el desempe\~no de detectores $100\%$ eficientes que observan el mismo rango angular que los detectores de radio.
Se encontr\'o que para mejorar el desempe\~no del detector aqu\'i estudiado una disminuci\'on en el nivel de disparo local de cada antena entregar\'ia la mayor ganancia de exposici\'on.
No s\'olo fue posible corroborar la factibilidad de la detecci\'on de neutrinos c\'osmicos mediante detectores de superficie conformado por antenas de radio, sino que adem\'as se comprob\'o que la t\'ecnica es competitiva frente a la siguiente generaci\'on de detectores de neutrinos, en estado de construcci\'on.