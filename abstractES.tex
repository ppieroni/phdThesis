\begin{titlepage}

% $\,$\\[2.5cm]
% \textbf{\Large  Medición del flujo de neutrinos cósmicos ultra energéticos mediante detectores de superficie}
% $\,$\\[2.5cm]

\noindent
\textbf{Resumen: }

Esta tesis estudia la medici\'on de neutrinos cósmicos ultra energéticos mediante detectores de superficie.
B\'asicamente existen dos mecanismos a trav\'es de los cuales los neutrinos en el rango del EeV pueden inducir señales distinguibles a nivel de superficie.
El primero consiste en la interacción de un neutrino en la atmósfera, via corrientes cargadas o neutras, y la subsiguiente producci\'on de una cascada atmosférica extendida descendente.
El segundo se basa en la interacción de un neutrino tau en la corteza terrestre al atravesarla de manera rasante.
Este proceso da lugar a un leptón tau capaz de escapar hacia la atmósfera e iniciar en su decaimiento una lluvia extendida ascendente de alto \'angulo.
En ambos casos, cuando la cascada se inicia cerca del detector, existen características en la señal producida, relacionadas con la presencia de componente electromagnética en la lluvia, que permiten discriminar este tipo de eventos del fondo dominante debido a rayos cósmicos de origen hadrónico.

Este trabajo se divide en dos partes, que abordan dos tecnolog\'ias diferentes para estudiar las lluvias atmosf\'ericas producidas por neutrinos.
La primera parte trata la b\'usqueda de neutrinos basada en la interacci\'on de las part\'iculas de la lluvia con el Detector de Superficie del Observatorio Pierre Auger.
En particular se describe el procedimiento de identificaci\'on, las posibles fuentes de fondo, el método desarrollado para calcular la exposición y sus incertezas sistemáticas asociadas en la b\'usqueda.
Asimismo, se presenta por primera vez la integración de los distintos análisis complementarios desarrollados en Auger, que incluyen dos criterios de reconocimiento de neutrinos descendentes y uno de neutrinos rasantes, en un único resultado, as\'i como la inclusión del envejecimiento del detector en el cálculo de la exposición.
Ningún candidato a neutrino fue encontrado en los datos adquiridos entre el 1 de enero de 2004 y el 20 de junio de 2013. Suponiendo un flujo diferencial típico $\Phi(E_\nu) = k\cdot E_\nu^{-2}$ y una relación entre sabores de $1:1:1$, se fijó un límite sobre el flujo de neutrinos de cada sabor, de $k < 6.4 \times 10^{-9}~{\rm GeV~cm^{-2}~s^{-1}~sr^{-1}}$ con un nivel de confianza del 90\% en el rango de ${1.0 \times 10^{17}}$ {eV} - ${2.5 \times 10^{19}}$ {eV}, el m\'as estricto hasta la fecha.

En la segunda parte de este trabajo se estudia el potencial de la detecci\'on de las ondas de radio producidas por lluvias atmosf\'ericas para identificar aquellas producidas por neutrinos, y se investigan las capacidades y limitaciones de un arreglo de antenas de radio a la hora de detectar neutrinos ultra energéticos rasantes. 
Para ello se utilizaron simulaciones que incluyen la interacción del neutrino en la Tierra, el decaimiento del lept\'on tau en la atmósfera, el subsiguiente desarrollo de la lluvia atmosférica extendida con producci\'on de radiaci\'on electromagn\'etica en frecuencias de radio, y la generaci\'on de la señal en las antenas del detector.
A partir de estas simulaciones se desarrollaron algoritmos de reconstrucción e identificación que permiten distinguir eventos iniciados por neutrinos de los generados por el fondo de lluvias hadrónicas.
Se utilizaron estos criterios para calcular la exposición que podrían alcanzar detectores de diferentes tamaños y topologías, concluyendo que la técnica puede ser competitiva para la siguiente generación de detectores de neutrinos.
\\[0.2cm]

\noindent
\textbf{Palabras claves: } rayos cósmicos, neutrinos UHE, neutrinos cosmogénicos, Observatorio Pierre Auger, cascadas atmosf\'ericas, emisión y detección de ondas de radio.\\

\end{titlepage}
