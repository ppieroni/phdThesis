\begin{titlepage}

$\,$\\[2.5cm]
\textbf{\Large  Medición del flujo de neutrinos cósmicos ultra energéticos mediante detectores de superficie}\\[2.5cm]

\noindent
\textbf{Resumen: }
los detectores de superficie son sensibles al flujo de neutrinos cósmicos ultra energéticos en el rango del EeV. 
Su detección se basa en dos mecanismos, mediante los cuales los neutrinos pueden dejan una señal distinguible en el detector.
El primero consiste en en la interacción de un neutrino en la atmósfera, via corrientes cargadas o neutras, capaz de iniciar una cascada atmosférica extendida.
El segundo se basa en la interacción de un neutrino tau que atravieza la corteza terrestre de manera rasante. 
Como producto de esta interacción puede producirce un leptón tau capaz de escapar hacia la atmósfera e iniciar en su decaimiento una lluvia extendida ascendente.
En ambos casos, cuando la cascada se inicia muy cerca del detector, existen características en la señal producida, relacionadas con la precencia de la componente electromagnética de la lluvia, que permiten discriminar este tipo de eventos del fondo provocado por rayos cósmicos de origen hadrónico.

En la primer parte de esta tesis se describe el procedimiento de búsqueda, las posibles fuentes de fondo, el método desarrollado para calcular la exposición y sus incertezas sistemáticas asociadas, al detectar neutrinos ultra energéticos con el Detector de Superficie del Observatorio Pierre Auger.
En particular se presenta por primera vez la integración de los tres análisis vigentes en el observatorio, que incluyen dos criterios de reconocimiento de neutrinos descendentes y uno para neutrinos rasantes, en un único resultado y la inclusión del envejecimiento del detector en el cálculo de la exposición.
Ningún candidato a neutrino fue encontrado en los datos adquiridos entre 1-Ene-2004 hasta 20-Jun-2013. Asumiendo un flujo diferencial típico $\Phi(E_\nu) = k\cdot E_\nu^{-2}$ y una relación entre sabores de $1:1:1$, se fijó un límite sobre el flujo de neutrinos de cada sabor, de $k < 6.4 \times 10^{-9}~{\rm GeV~cm^{-2}~s^{-1}~sr^{-1}}$ con un nivel de confianza del 90\% en el rango de \cant{1.0 \times 10^{17}}{eV} - \cant{2.5 \times 10^{19}}{eV}.

En la segunda parte de este trabajo estudian por primera ves las capacidades y limitaciones de un arreglo de antenas de radio a la hora de detectar neutrinos ultra energéticos rasantes.
Para ello se utilizaron simulaciones Montecarlo que incluyen la interacción del neutrino en la tierra, su decaimiento en la atmósfera, el subsiguiente desarrollo de la lluvia atmosférica extendida y la simulación de la señal generada en las antenas del detector.
A partir de estas simulaciones se desarrollaron técnicas simples de reconstrucción e identificación que permiten distinguir eventos iniciados por neutrinos de los generados por él fondo de lluvias hadrónicas. 
Se utilizaron estos criterios para calcular la exposición que podrían alcanzar detectores de diferentes tamaños y topologías, concluyendo que la técnica puede ser competitiva con la siguiente generación de detectores de neutrinos.
\\[0.2cm]

\noindent
\textbf{Palabras claves: } rayos cósmicos y neutrinos UHE, neutrinos cosmogénicos, Observatorio Pierre Auger, emisión y detección radio.\\

\end{titlepage}