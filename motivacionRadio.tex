\chapter{Motivaci\'on}
\label{ch:motRadio}

Si bien la emisi\'on de radio de las EAS fue observada por primera vez por Jelley en 1964 \cite{jelley1966radio} y su investigaci\'on fue bastante activa durante los a\~nos 70 \cite{allan1971progress}, no fue sino hasta las \'ultimas d\'ecadas que desarrollos en electr\'onica de alta velocidad y en teor\'ia de la informaci\'on renovaron el inter\'es en el \'area.
Como consecuencia de este resurgimiento se han impulsado varios experimentos que buscan estudiar el potencial de la t\'ecnica, como por ejemplo CODALEMA \cite{ardouin2005radio}, LOPES \cite{huege2012lopes}, LOFAR \cite{horandel2009lofar}, Tunka-Rex \cite{schroder2013tunka} o AERA \cite{kelley2011aera}, la extensi\'on de radio del observatorio Pierre Auger.
Asimismo esto produjo la necesidad de desarrollar nuevas t\'ecnicas de calculo anal\'iticas \cite{huege2003radio,scholten2008macroscopic}, m\'etodos de Monte Carlo \cite{huege2007monte,ludwig2011reas3} o m\'etodos semi-anal\'iticos\cite{scholten2009macroscopic}.

El interés en la técnica reside en que la amplitud del pulso de radio generado en las antenas se encuentra correlacionado con el número de partículas y la energía de la cascada electromagnética, lo que provee información calorimétrica sobre el primario, de manera similar a la técnica de fluorescencia.
Adem\'as, los pulsos almacenados por un arreglo de antenas guardan información sobre el desarrollo longitudinal de la cascada y su $X_{max}$, es decir, son sensibles a la composici\'on de los rayos c\'osmicos primarios \cite{cite:hauge_rec,cite:lofar_rec}.
Finalmente, los arreglos de antenas de radio comparten con los de detectores de partículas sus dos principales ventajas, la reconstrucción de la dirección de arribo se realiza a partir de la medición de tiempos de disparo y su ciclo de trabajo es cercano al \cant{100}{\%}\footnote{por ejemplo, los períodos en los que haya tormentas eléctricas deben descartarse.}.
Todas estas características, sumadas al bajo costo que posee cada antena respecto a los detectores convencionales de partículas hacen de esta técnica de detección una opción competitiva para la próxima generación de detectores de rayos c\'osmicos ultra energ\'eticos.

La segunda parte de esta tesis corresponde al estudio de las capacidades y limitaciones de un arreglo de superficie de 100000 antenas de radio, desplegadas sobre una superficie de al rededor de \cant{\sim60000}{km^2}, a la hora de detectar neutrinos cósmicos ultra energéticos.
Estos números surgen de que la siguiente generación de detectores de superficie requerir\'a aumentar el área cubierta y la cantidad de estaciones respecto del de Auger (que posee el detector de superficie de punta), y al mismo tiempo, mantener la granularidad (distancia entre detectores) con el fin de conservar el nivel de eficiencia a baja energ\'ia (\cant{\sim 10^{17}}{eV}).
Actualmente existe un proyecto llamado Giant Radio Array for Neutrino Detection (GRAND)~\cite{cite:grand_prop}, que planea la instalaci\'on de 90000 antenas en un arreglo cuadrado de \cant{800}{m} de paso sobre una superficie de \cant{60000}{km^2} en la cordillera de Tianshan \cite{cite:grand_tec} con el fin de detectar neutrinos c\'osmicos ultra energ\'eticos.

\vspace*{5mm}
\textbf{AGREGAR ALGO DE DETALLE DE GRAND?}
\vspace*{5mm}
% El costo aproximado sin deployment y por tanque en Auger es de $\sim$\EUR{8000} \cite{augerDesignReport} mientras que por antena rondaría los \EUR{600}.
% Por otro lado, esta ganancia de un factor $\sim13$ en el costo de cada detector se ve compensada por la mayor cantidad de recursos necesarios para despliegue y mantenimiento, que aumenta con la superficie instrumentada y la cantidad de detectores instaldos.
% Con todo esto, 10000 antenas de radio disminuirían en un factor $\sim2$ el costo de los detectores respecto de los 1600 tanques de Auger, dejando suficiente margen para el despliegue y mantenimiento sobre un área que, si bien depende de la topografia del arreglo, rondaría los \cant{200\times200}{km^2}.

En el capítulo \ref{ch:easRadio} se realiza una introducción a la emisión de radio de las lluvias atmosféricas en la que se discutirán sus características principales, sus mecanismos de producción y se discutirá un modelo simplificado que mediante hipotesis geom\'etricas permite recuperar sus características principales.
Más adelante, el capítulo \ref{ch:simulacionRadio} contiene una revisión de \zhs{}, el programa utilizado para simular la emisión de radio de las EAS, una caracterización detallada de las señales a nivel del suelo que generan los neutrinos ES y posibles estrategias para separar eventos generados por neutrinos ES de los producidos por lluvias protonicas horizontales.
Finalmente en el capítulo \ref{ch:resultadosRadio} se calcula la exposición que podría alcanzar un detector de superficie de 90000 antenas de radio, considerando diferentes topografías.


