\chapter{Introducci\'on}
\label{ch:intro}

La f\'isica de astropart\'iculas experimenta en la actualidad un desarrollo sin precedentes. 
El mensajero tradicional del cielo, el fot\'on, ha sido complementado a principios del siglo XX mediante la observaci\'on de part\'iculas cargadas (rayos c\'osmicos) y, durante las \'ultimas d\'ecadas se han realizado esfuerzos en el desarrollo de la astrof\'isica de neutrinos.
Todos estos mensajeros acarrean informaci\'on sobre la fuente que los produjo, lo que los convierte en nuestra ventana de acceso al cosmos.

El desarrollo de la astronom\'ia de neutrinos lleva varias d\'ecadas de desarrollo. 
El 23 de febrero de 1987 Kamiokande II recibi\'o la se\~nal de 11 neutrinos con energ\'ias en el \'orden del MeV, mientras que simult\'aneamente el detector IBM observ\'o 8.
Estos eventos, en coincidencia con la observaci\'on de la supernova SN 1987A, dieron lugar a la primera detecci\'on de neutrinos provenientes de una fuente extra gal\'actica.
Ya en esta d\'ecada, IceCube ha realizando las primeras mediciones de neutrinos c\'osmicos en energ\'ias del PeV~\cite{cite:IceCube1}.
Estos descubrimientos admiten una nueva mirada al universo, expandiendo as\'i las posibilidades de observaci\'on.
Por un lado los rayos c\'osmicos cargados son deflectados debido a los campos magn\'eticos intergal\'acticos, mientras que por otro los fotones son absorbidos en las zonas opacas del espacio. 
Sin embargo, los neutrinos no sufren ninguna de estas alteraciones ya que no poseen carga el\'ectrica y adem\'as, como s\'olo interact\'uan mediante fuerza d\'ebil y su secci\'on eficaz es peque\~na, no son retenidos en las zonas densas del universo.
Esta caracter\'istica, que les provee muy buenas cualidades a la hora de trasladar informaci\'on de un punto a otro del cosmos, los hace extremadamente dif\'iciles de detectar en la Tierra, lo que representa un desaf\'io muy interesante de abordar.
% Existe una gran cantidad de rese\~nas sobre el estado del \'area, como por ejemplo \cite{XXX}, en las que se remarca el potencial de utilizar el neutrino como mensajero c\'osmico.

En esta Tesis se investiga la detecci\'on de neutrinos c\'osmicos ultra energ\'eticos mediante detectores de superficie.  En el cap\'itulo \ref{ch:NeutrinosUHE} se revisa la situaci\'on te\'orica y experimental del \'area. La Tesis se subdivide luego en dos partes, correspondiendo a las dos líneas de trabajo realizadas (abordadas?, emprendidas?).

La primera parte trata la medici\'on del flujo difuso de neutrinos en el rango energ\'etico de \cant{10^{17}}{eV} a \cant{10^{20}}{eV} con el detector de superficie del Observatorio Pierre Auger. En el cap\'itulo \ref{ch:detectorAuger} se reveen primero las caracter\'isticas del detector m\'as relevantes para este análisis.
El cap\'itulo \ref{ch:easAuger} discute en detalle la identificación de neutrinos a través de lluvias atmosf\'ericas extendidas, y se presenta las tres estrategia de búsqueda de neutrinos encaradas por Auger. La simulación de eventos de neutrinos se describe en el cap\'itulo \ref{ch:simulacionAuger}, mientras que la reconstrucci\'on de lluvias inclinadas y las selecci\'on de los eventos iniciados por neutrinos se trata en el cap\'itulo \ref{ch:selAuger}.
En el cap\'itulo \ref{ch:resAuger} se describe la combinación de los tres análisis en el c\'alculo de la exposici\'on del Observatorio, lo que permite traducir la búsqueda de eventos en una medición del flujo de neutrinos. Por último, el Capítulo XXX presenta los resultados obtenidos y la comparación con otros experimentos y con predicciones te\'oricas. Como toda medición realizada en el marco de un gran colaboración, los resultados presentados son el fruto del esfuerzo conjunto de muchas individuos. Mi contribución particular fue el desarrollo del análisis de neutrinos correspondiente a incidencia rasante, en estrecha colaboración con el Dr. Yann Guardincerri, y la combinación de los tres análisis para obtener un resultado global del experimento.

La segunda parte de la Tesis se enfoca en la detecci\'on de neutrinos c\'osmicos ultra energ\'eticos a partir de la radiaci\'on electromagn\'etica producida por la lluvia atmosf\'erica.  En particular, se estudia el desempe\~no en la medici\'on del flujo que podr\'ia alcanzar un detector de superficie conformado por 90000 antenas de radio.
En el cap\'itulo \ref{ch:motRadio} se motiva el estudio de este tipo de detectores, mientras que en el cap\'itulo \ref{ch:easRadio}, y como complemento al cap\'itulo \ref{ch:easAuger}, se revisa la emisi\'on de ondas de radio de las lluvias atmosf\'ericas extendidas.
Los m\'etodos empleados en la simulaci\'on de la se\~nal sobre el detector se detallan en el cap\'itulo \ref{ch:simulacionRadio} y su caracterizaci\'on en el cap\'itulo \ref{ch:caracterizacionRadio}.
Finalmente, en el cap\'itulo \ref{ch:resultadosRadio} se aborda tanto la factibilidad de la detecci\'on como el c\'alculo de la exposici\'on en un arreglo de antenas de radio. El material presentado en esta segunda parte corresponde esencialmente en su totalidad a desarrollos realizados en el marco de esta Tesis.

