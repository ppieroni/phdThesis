\chapter{Introducci\'on}
\label{ch:intro}

La f\'isica de astropart\'iculas experimenta actualmente un desarrollo muy veloz. 
El mensajero tradicional del cielo, el foton, ha sido complementado a principios del siglo XX con la observaci\'on de part\'iculas cargadas (rayos c\'osmicos) y, durante las \'ultimas d\'ecadas, mediante el desarrollo de la astrof\'isica de neutrinos.
Todos estos mensajeros, ricos en informaci\'on, nos permiten estudiar las propiedades de fuentes astrof\'isicas en todo el cosmos.

La astronom\'ia de neutrinos se encuentra en sus comienzos. 
Esta admite una nueva mirada al universo expandiendo las posibilidades de observaci\'on.
Los rayos c\'osmicos cargados son deflectados debido a los campos magn\'eticos intergal\'acticos mientras que los rayos gamma son absorbidos en zonas opacas del espacio. 
Sin embargo los neutrinos no sufren ninguna de estas alteraciones ya que no poseen carga el\'ectrica y adem\'as s\'olo interact\'uan mediante fuerza d\'ebil y su secci\'on efic\'az es baja.
Sin embargo, esta cualidad los hace extremadamente dif\'iciles de detectar en al tierra, lo que representa un desaf\'io muy interesante de abordar.
Existe una gran cantidad de rese\~nas sobre el estado del \'area, como por ejemplo \cite{XXX}, en las que se remarca el potencial de utilizar el neutrino como mensajero c\'osmico.

En este trabajo se aborda la detecci\'on de neutrinos c\'osmicos ultra energ\'eticos mediante detectores de superficie.
En la primer parte de esta tesis se presenta la medici\'on del flujo de neutrinos c\'osmicos en el rango energ\'etico de \cant{10^{17}}{eV} y \cant{10^{20}}{eV} con el detector de superficie del Observatorio Pierre Auger.
El cap\'itulo \ref{ch:easAuger} contiene una introducci\'on a la f\'isica de las lluvias atmosf\'ericas extendidas que generan los rayos c\'osmicos ultra energ\'eticos. 
En el cap\'itulo \ref{ch:detectorAuger} se realiza un recuento de las caracter\'isticas m\'as importantes del detector de superficie del Observatorio Pierre Auger.
M\'as adelante, en el ca\'itulo \ref{ch:estrategiaAuger} se detalla la estrategia utilizada en el experimento para medir el flujo de neutrinos c\'osmicos ultra energ\'eticos.
El proceso de simulaci\'on de la se\~nal esperada sobre el detector se aborda en el cap\'itulo \ref{ch:simulacionAuger}, mientras que la reconstrucci\'on y selecci\'on de los eventos iniciados por neutrinos se trata en el cap\'itulo \ref{ch:selAuger}.
Por \'ultimo en el cap\'itulo \ref{ch:resAuger} se trata el c\'alculo de la exposici\'on del observatorio al flujo, la b\'usqueda de candidatos y la comparaci\'on de los resultados con las predicciones te\'oricas.
Por otro lado, en la seunda parte de la tesis se estudia el desempe\~no que podr\'ia alcanzar un detector de superficie conformado por 90000 antenas de radio al medir el mismo flujo de neutrinos.
En el cap\'itulo \ref{ch:motRadio} se exponen brevemente los motivos por los que vale la pena el estudio de este tipo de detectores.
El cap\'itulo \ref{ch:easRadio}, como complemento al cap\'itulo \ref{ch:easAuger} se estudia la emisi\'on de ondas de radio de lluvias atmosf\'ericas extendidas.
Los m\'etodos empleados en la simulaci\'on de la se\~nal sobre el detector se detalla en el cap\'itulo \ref{ch:simulacionRadio} y su caracterizaci\'on en el cap\'itulo \ref{ch:caracterizacionRadio}.
Por \'ultimo, en el cap\'itulo \ref{ch:resultadosRadio} se aborda tanto la factibilidad de la detecci\'on como el c\'alculo de la exposici\'on en un arreglo de antenas de radio.

\section{Importancia de la detecci\'on de neutrinos c\'osmicos}
%
%
%%
%%%
% 	VER EL PROPOSAL DE GRAND!
%%%
%%
%
%
El estudio de rayos c\'osmicos ultra energ\'eticos (UHECRs por sus siglas en ingl\'es) ha estimulado en gran medida la actividad experimental y te\'orica en el campo de la astrof\'isica.
Aunque su espectro de energ\'ia ha sido caracterizado en un rango sorprendente, que abarca 14 \'ordenes de magnitud, quedan muchos misterios por resolver, entre ellos su origen y sus mecanismos de aceleraci\'on.
En esta direcci\'on, la medici\'on de UHECRs cargados presenta dos grandes limitaciones, su deflexi\'on en los campos magn\'eticos y lo que se conoce como el corte GZK.
A energ\'ias por debajo de los \cant{10^{19.5}}{eV} las trayectorias desde la fuente se ven modificadas debido a la interacci\'on con los campos magn\'eticos gal\'acticos e intergal\'acticos lo que implica que la direcci\'on de arribo de los rayos a la tierra no apunta a la fuente.
Por otro lado, el corte GZK refiere al mecanismo propuesto por Greisen, Zatsepin y Kusmin~\cite{cite:Greisen,cite:Zatsepin}, que provoca una caida en el flujo de UHECRs por encima de \cant{5\times10^{19}}{eV} debido a la p\'erdida de energ\'ia inducida por la interacci\'on con el fondo c\'osmico de microondas (CMB por sus siglas en ingl\'es), via la siguiente reacci\'on:
%
\begin{equation}
p + \gamma_{\rm CMB}\, \rightarrow\, \Delta^{+}(1232)  \rightarrow p +\pi^{0}\quad {\rm or}\quad n +\pi^{+}
\end{equation}
%
La longitud de atenuaci\'on para este proceso es $L_{att}=\frac{L_{int}}{y}$, donde $y$ es la fracci\'on de energ\'ia perdida por longitud de interacci\'on y $L_{int}$ es la longitud de interacci\'on, dada por $L_{int}=(\sigma_{p\gamma}\times n_{\gamma})^{-1}$.
Valores t\'ipicos son $\sigma_{p\gamma}\sim 10^{-28}$~cm$^{2}$, $ n_{\gamma}=410\,{\rm cm}^{-3}$ and $y\sim0.5$\footnote{$y\sim0.2$ a la energ\'ia de corte y se incrementa hasta 0.5.}, resultando en 
% \begin{equation}
$L_{att}=(\sigma_{p\gamma}\times n_{\gamma}\times y)^{-1}\sim 15\mbox{ }{\rm Mpc}$. 
Ya que a estas energ\'ias los rayos c\'osmicos son mayormente extra gal\'acticos, el corte GZK limita la m\'axima energ\'ia que puede ser observada en la tierra y provocando una supresi\'on del flujo por encima de los \cant{50}{EeV}.
En la figura \ref{fig:protProp} se muestra la longitud $L_{att}$ como funci\'on de la energ\'ia para protones y n\'ucleos de hierro. 
Es posible observar como a partir de los \cant{50}{EeV} esta cantidad decae hasta un tama\~no inferior al del Super cluster de Virgo a los \cant{1000}{EeV}.
%
\begin{figure}[ht]
	\begin{center}
	\includegraphics[width=0.55\textwidth]{fig/introduccion/proton_propaga_espanol}
	\caption{\label{fig:protProp} Longitud de atenuaci\'on como funci\'on de la energ\'ia para protones y n\'ucleos de hierro. Se observa que a partir de \cant{50}{EeV} esta cantidad decae hasta el tama\~no del Super cluster de Virgo a los \cant{1000}{EeV}.}
	\end{center}
\end{figure}
%

Por otro lado, el observatorio Pierre Auger ha medido el flujo de rayos c\'osmicos combinando un detector de superficie con t\'ecnicas de fluorescencia y ha acumular suficiente estad\'istica para medir el flujo con presici\'on hasta alrededor de \cant{2\times10^{20}}{eV}. 
Tambi\'en pudo corroborar la supresi\'on del flujo para energ\'ias superiores a los \cant{10^{19.6}}{eV}~\cite{cite:AugerSpectrum}, tal como se observa en la figura \ref{fig:specGZK}.
%
\begin{figure}[ht]
	\begin{center}
	\includegraphics[width=\textwidth]{fig/introduccion/spectrum_withGZK}
	\caption{\label{fig:specGZK} Espectro de UHECRs medido con el observatorio Pierre Auger. En l\'inea punteada se muestra el ajuste por leyes de potencia partidas y en l\'inea llena la dos leyes de potencia y una funci\'on suave. Las barras corresponden al error estad\'istico de cada punto, mientras que el error sistem\'atico representa el $22\%$ de la energ\'ia.}
	\end{center}
\end{figure}

De manera similar, el flujo de fotones ultra energ\'eticos, por encima de \cant{\sim 10^{14}}{eV}, no puede ser de naturaleza extra gal\'actica, debido a la producci\'on de pares en la interacci\'on con fotones del fondo de microondas, seg\'un\cite{cite:photonInt1,cite:photonInt2}:
%
\begin{equation}
\gamma_{UHE} + \gamma_{CMB} \rightarrow e^- + e^+
\end{equation}
%
En la figura \ref{fig:photProp} se grfica la longitud de atenuaci\'on de los fotones como funci\'on de la energ\'ia\footnote{En este caso, la longitud de atenuaci\'on corresponde con la distancia necesaria para que el flujo se reduca a la mitad.}
%
\begin{figure}[ht]
	\begin{center}
	\includegraphics[width=0.55\textwidth]{fig/introduccion/photon_propaga_espanol}
	\caption{\label{fig:photProp} Longitud de atenuaci\'on para fotones. Los $\gamma$ con energ\'ias entre \cant{10^{14}}{eV} y \cant{10^{18}}{eV} pr\'acticamente no pueden alcanzar la tierra desde distancias mayores a \cant{1}{Mpc}. Las etiquetas IR, CMB y URB (ver texto) corresponden al fondo dominante contra el que interact\'uan los $\gamma_{UHE}$.}
	\end{center}
\end{figure}
%
Dependiendo de la energ\'ia, los $\gamma_{UHE}$ pueden interactuar tambi\'en con el fondo de radiaci\'on infraroja (IR)\cite{cite:IR} y con el fondo de radio universal (URB)\cite{cite:URB}.

Como consecuencia el tercer mensajero $-$ el neutrino $-$ cobra una importancia adicional, lo que convirti\'o su detecci\'on en uno de los mayores logros de la astrof\'isica contempor\'anea.
Esto se debe a que los neutrinos no sufren ninguna de las desventajas mensionadas hasta el momento. 
Debido a que interact\'uan mediante interacci\'on d\'ebil y a que su secci\'on efic\'az resulta extremadamente peque\~na, pueden viajar distancias cosmol\'ogicas e incluso escapar de la regi\'on en la que fueron producidos casi sin p\'erdidas de energ\'ia.
Por otro lado, debido a que son el\'ectricamente neutros su trayectoria no se ver\'a deflectada debido a la interacci\'on con los campos magn\'eticos intra y extra gal\'acticos.
Por este motivo, la direcci\'on de arribo de los neutrinos c\'osmicos detectados guardar\'a completamente la informaci\'on del lugar del universo en el que fueron producidos.
Por estos motivos, los neutrinos representan una opci\'on \'unica, que permite detectar directamente posibles fuentes de UHECRs.

En las siguientes secciones de este cap\'itulo se presentar\'a una discusi\'on sobre las posibles fuentes y flujos de neutrinos c\'osmicos ultra energ\'eticos, a saber, el mecanismo GZK, n\'ucleos de galaxias activos (AGN) y explosiones de rayos gamma (GRB).
Tambi\'en se realizar\'a un recuento de los ezfuersos experimentales pasados, presentes y futuros en este campo.

\section{Posibles fuentes y flujos esperados}

Existen varios modelos en la literatura que predicen flujos de neutrinos c\'osmicos ultra energ\'eticos.
La la supresi\'on observada en el flujo por encima de los \cant{50}{EeV} refuerza la idea de la existencia de un flujo difuso de neutrinos cosmog\'enicos.
En este caso, estos son producidos durante la propagaci\'on de un UHECR a trav\'es del universo. 
Adem\'as pueden ser producidos en la aceleraci\'on de protones y nucleos en n\'ucleos de galaxias activos (o AGN por sus siglas en ingl\'es)\cite{cite:nuAGN} o por producci\'on de fotopiones en explosiones de rayos gamma (GRB por sus siglas en ingles)\cite{cite:nuGRB}.
Detectores como AMANDA o IceCube, su sucesor, se encuentran bien posicionados para realizar b\'usquedas de fuentes con espectros de ley de potencia fuertes ($\propto E^{-2}$) en el rango que comprende desde el TeV  hasta el PeV. 
Para fuentes cuyo flujo presenta un m\'aximo en energ\'ias por encima de los \cant{100}{PeV} el flujo predicho resulta ser peque\~no, requiriendo detectores con mayor exposici\'on.

Otros posibles mecanismos de producci\'on de neutrinos c\'osmicos estan relacionados con el decaimiento de part\'iculas ex\'oticas extremadamente masivas tales como defectos topol\'ogicos\cite{cite:nuTopDefects}, o con la interacci\'on de neutrinos energ\'eticos con el fondo de neutrinos del Big-Bang via la resonancia Z-burst\cite{cite:nuZBurst_init}.
Estos flujos, que pretend\'ian explicar el origen de los rayos c\'osmicos ultra energ\'eticos, han sido fuertemente constre\~nidos por los experimentos m\'as recientes\cite{cite:nuConstraintsTD}.
En las siguientes secciones se desarrollas las ideas enumeradas hasta aqu\'i.

	\subsection{Neutrinos GZK}
	\label{sbsc:introGZK}
	Greisen, Zatsepin and Kusmin propusieron que los rayos c\'osmicos cuya energ\'ia supere los \cant{5\times10^{19}}{eV}, al propagarse por el universo interactuar\'an con el fondo de microondas produciendo neutrinos, seg\'un la ecuaci\'on \ref{eq:pionDecay}.
	\begin{equation}\label{eq:pionDecay}
	\pi^{+} \rightarrow \mu^{+}+ \nu_{\mu} \rightarrow e^{+} + \nu_{e} + \bar{\nu}_{\mu} + \nu_{\mu}
	\end{equation}
	
	La precencia del corte GZK, indica que los UHECRs provienen de fuentes extra gal'acticas.
	Esto implica que los llamados neutrinos GZK son el flujo m'as verosimil entre todas las posibles teor'ias. 
	Sin embargo, su c'alculo contiene una gr\'an cantidad de supuestos que se traducen en incertezas en el resultado final.
	Los factores m\'as relevantes en su determinaci\'on son los siguientes~\cite{cite:nuEngel,cite:nuAve,cite:nuAhlers1,cite:nuAllard1,cite:nuYuksel}:
	
	\begin{description}
	\item[Composici\'on de los UHECRs] Las primeras predicciones sobre el flujo de neutrinos cosmol\'ogicos asum\'ian que los primarios son protones, mientras que recientemente han aparecido resultados en los que se consideran como primarios $^{56}{\rm Fe}$, $^{4}{\rm H}$, $^{16}{\rm O}$ o mezclas entre ellos y con protones\cite{cite:nuAve,cite:nuHooper}.
	Los n\'ucleos m\'as pesados pierden energ\'ia por foto-desintegraci\'on, produciendo n\'ucleos secundarios, que luego producen fotopiones que al decaer generan neutrinos UHE$\nu$.
	Adem\'as, fujos de anti neutrinos electr\'onicos son predichos via decaimiento de neutrones\cite{cite:nuFeComposition}, pero su energ\'ia resulta peque\~na para ser detectados por Auger.
	Por otro lado, los flujos que provienen de una composici\'on primaria no pura son peque\~nos cuando se los compara con una composici\'on de protones pura\cite{cite:nuHooper}.
	En particular, la energ\'ia por nucle\'on luego de una fotodesintegraci\'on resulta mucho m\'as peque\~na que la del primario, lo que desfavorece la generaci\'on de neutrinos GZK a trav\'es de este proces.
	Tambi\'en existen modelos en los que se propone una distribuci\'on de primarios en acuerdo con la composici\'on observada en los rayos c\'osmicos gal\'acticos\cite{cite:nuAllard1}.
	Como consecuencia de estas suposiciones los flujos predichos pueden fluctuar en un \'orden de magnitud.
	 
	Resultados recientes de Auger indican, aunque con cierto nivel de debate, que el flujo de UHECRs se encuentra dominado por n\'ucleos pesados\cite{cite:augerComposition}.
	Sin embargo, mediciones en HiRes y Telescope Array\cite{cite:taComposition} han observado lo opuesto.
	Si se llegase a dar una observaci\'on por encima de las predicciones para primarios pesados, se podr\'ia echar luz sobre este asunto.
	 
	\item[Perfil de energ\'ia] El espectro energ\'etico de los rayos c\'osmicos en el punto de inyecci\'on se supone t\'ipicamente una ley de potencia con la siguiente dependencia en energ\'ia:
	%
	\begin{equation}
		\frac{dN}{dE}=P_{0} \times E^{-\alpha} \times \exp{(-\frac{E}{E_{c}})}
	\end{equation}
	%
	donde $P_0$ es una normalizaci\'on y el \'indice espectral $\alpha$ toma valores entre 1.8 y 2.7, siendo $\alpha\sim2.3$ el m\'as favorecido.
	El corte de energ\'ia en la inyecci\'on, $E_c$ es considerado entre \cant{10^{20}}{eV} y \cant{10^{23}}{eV}.
	Tanto $\alpha$ como $E_c$ dependen de las caracter\'isticas de la fuente y del mecanismo de aceleraci\'on de los rayos c\'osmicos.
	Una vez que estos valores son elegidos, la normalizaci\'on $P_0$ se utiliza para hacer coincidir el flujo con el observado experimentalmente en la Tierra.
	Valores grandes de $\alpha$ o peque\~nos de $E_c$ producen flujos de neutrinos m\'as peque\~nos a energ\'ias de observaci\'on de Auger debido a la disminuci\'on en la cantidad de protones de altas energ\'ia en la fuente.
	
	\item[Modelo cosmol\'ogico]
	La cosmolog\'ia del universo es otro factor que genera incerteza en la magnitud del fluho de neutrinos GZK. 
	En la actualidad, observaciones astrof\'isicas apuntan hacia modelos con una constante cosmol\'ogica $\Lambda$\cite{cite:Lambda}, en contraste con los c\'alculos de mediados de la d\'ecada del 90, que supon\'ian un universo de Einstein-de Sitter ($\Omega_{M}=1$). 
	El modelo favorecido actualmente posee $\Omega_{\Lambda}=0.7$ y $\Omega_{M}=0.3$\cite{cite:LambdaM}, lo que significa que la energ\'ia oscura representa alrededor del 70\% de la energ\'ia del universo.
	Como consecuencia el universo debi\'o haberse estado expandiendo mas lentamente en el momento en el que los flujos cosmol\'ogicos se generaron, ocasionando un incremento en los flujos esperados para corrimientos al rojo grandes.
	Engel et al. \cite{cite:nuEngel} compararon el flujo derivado de los dos modelos cosmol\'ogicos y encontraron que valores de $\Omega_{\Lambda}=0.7$ podr\'ian ocasionar hasta un 60\% de incremento en el flujo de neutrinos esperado.
	
	\item[Evoluci\'on cosmol\'ogica]
	La predicci\'on de los flujos de neutrinos dependen fuertemente de la evoluci\'on cosmol\'ogica de de las fuentes de rayos c\'osmicos.
	Existen al menos 4 modelso de evoluci\'on utilizados comunmente en la bibliograf\'ia:
	\begin{enumerate}
	 \item Sin evoluci\'on;
	 \item Star Formation Rate;
	 \item Active Galactic Nuclei-FRII (FRII) y
	 \item Strong Gamma Ray Burst (GRB).
	\end{enumerate}
	
	Las diferencias inducidas por esta elecci\'on puede modificar el flujo esperado en un \'orden de magnitud.
	
	\item[Secci\'on efic\'az proton-fot\'on]
	El ritmo de producci\'on de neutrinos mediante la interacci\'on:
	%
	\begin{displaymath}
	p + \gamma_{\rm CMB}\, \rightarrow\, \Delta^{+}  \rightarrow p +\pi^{0}\quad {\rm or}\quad n +\pi^{+}
	\end{displaymath}
	%
	viene dado por la secci\'on efic\'az prot\'on-fot\'on, $\sigma_{p\gamma}$.
	Las mediciones de esta cantidad provienen de aceleradores, por lo que se encuentran bien caracterizadas en un rango de energ\'ias totalmente distinto al que corresponde da lugar a la producci\'on de neutrinos.
	\item[Oscilaciones de neutrinos]
	En el decaimiento de un pi\'on, la relaci\'on con la que se producen neutrinos mu\'onicos y electr\'onicos es $2:1$. 
	En este trabajo, al menos que se especifique lo contrario, hablar de neutrinos de cierto sabor $\nu_x$, har\'a referencia al par $\nu_x + \bar\nu_x$.
	Dado que los neutrinos oscilan, al recorrer distancias cosmol\'igicas la relaci\'on entre los flujos $\nu_e :\nu_\mu :\nu_\tau$ de $1:2:0$ en la fuente se transformar\'a en una relaci\'on $1:1:1$ en la tierra.
	\end{description}
	%
	Aunque la existencia de los neutrinos cosmol\'ogicos es verosimil, los flujos predichos por este mecanismo var\'ian cuatro \'ordenes de magnitud.
	Algunas de estas predicciones se muestran en la figura \ref{fig:flujosGZK} para los tres sabores de neutrinos.
	
	\begin{figure}[ht]
		\begin{center}
		\includegraphics[width=\textwidth]{fig/introduccion/gzk_fluxes}
		\caption{\label{fig:flujosGZK} Flujos cosmog\'enicos para los tres sabores de neutrinos. En todos los casos el modelo cosmol\'ogico usado corresponde con $\Omega_{\Lambda}=0.7$ y $\Omega_{M}=0.3$. 
		En rojo se muestra un flujo GZK t\'ipico para una composici\'on pura de protones, un \'indice $\alpha$ = 2 y \cant{E_{c}=10^{21.5}}{eV}\cite{cite:nuEngel}.
% 		En linea azul s\'olida
% In solid light blue the GRB model for the cosmological evoultion is used\cite{cite:nuYuksel}.
% % In solid green Ref.\cite{cite:nuAhlers1} where they make a prediction using also the measurements of the Fermi-LAT experiment.
% In solid green a prediction using the measurements of the Fermi-LAT experiment\cite{cite:nuAhlers1}.
% The shaded area is obtained from Ref.\cite{cite:nuAllard1}, bracketing a wide range a parameters: 
% several source evolution models (not including uniform and FRII), 
% for pure protons and a mixed Galactic composition. 
% Including the uniform source evolution would lower the prediction by almost an order of magnitude.
% The pink dot-dashed line corresponds to an optimistic scenario with a FRII strong source evolution case with a pure proton composition, 
% % dip transition model 
% and $E_{c}=10^{21.5}$~eV\cite{cite:nuAllard1}. 
% The blue dashed lineis an extreme pessimistic scenario with pure
% iron composition and uniform evolution\cite{cite:nuAllard1}.
% % $E_{c}$ = 1021.5 eV. For the cosmological evolution of the cosmic ray sources, ${\cal H}$(z), they use the
% % aforementioned QSO model with the parameterization of Ref. [33], as given in Eqn.1.7. The
% % default cosmological assumption used is the model with
% 		
		}
		\end{center}
	\end{figure}
	
	\subsection{AGNs y GRBs}
	Los n\'ucleos de galaxias activos (AGN) pueblan de manera isotr\'opica el cielo y representan algunos de los obj\'etos m\'as luminosos en el espacio, con emisiones en el \'orden de los \cant{10^{45\pm3}}{erg/s}\cite{cite:GHZ}.
	Por este motivo no s\'olo son considerados unos de los m\'as fiables candidatos a fuentes de UHECRs, sino que varios autores predicen flujos medibles de neutrinos si la regi\'on de aceleraci\'on se encuentra rodeada de la suficiente cantidad de materia.
	
	Se cree que la enorme radiaci\'on emitida por los AGNs se alimenta de la energ\'ia gravitacional liberada por la materia que cae hacia su centro, en el que se ubica un agujero negro super masivo.
	Durante este proceso, el momento angular provoca que parte del material forme un disco de acresi\'on mientras que otra parte se vea despedida hacia chorros perpendiculares al mismo.
	En tal proceso, los shocks turbulentos aceleran part\'iculas hacia altas energ\'ias.
	De esta manera, una parte significativa de la energ\'ia gravitatoria es transferida a part\'iculas ultra relativistas, via el mecanismo de fermi de primer orden\cite{cite:Fermi1}.
	Por otro lado, la fricci\'on convierte la materia en caida en plasma, que produce un fuerte campo magn\'etico.
	Las colisiones entre los protones ultra relativistas que el intenso campo de fotones del AGN produce neutrinos de alta energ\'ia mediante $p\gamma \rightarrow \pi^{+} + n$ y el subsecuente decaimiento $\pi^{+} \rightarrow \mu^{+} + \nu_{\mu}$ deguido de $\mu^{+} \rightarrow e^{+} + \nu_{e} + \bar{\nu_{\mu}}$.
	
	Otros modelos proponen que los protones colisionan con part\'iculas de gas y polvo, prvocando $pp\,\rightarrow\,\pi s + X$.
	Como en el mecanismo GZK, los neutrinos incluidos inicialmente son $\nu_\mu$ y $\nu_e$ pero las oscilaciones homogeneizan los flujos entre sabores al llegar a la tierra.
	
	Dependiendo del lugar en el que la producci\'on de neutrinos se leva a cabo, existen dos tipos de modelos de producci\'on en AGNs: AGN core y AGN jet.
	En el primer caso, propuesto inicialmente por Stecker et. al.\cite{cite:nuAGN}, los protones acelerados interact\'uan con el campo de fotones dentro del n\'ucleo del AGN. 
	En el segundo modelo, dos jets relativistas emitidos perpendicularmente al disco de acresi\'on formando l\'obulos. 
	Los protones son acelerados mediante el mecanismo de fermi para luego interactuar con fotones del jet y producir neutrinos.
	Su flujo puede ser estimado midiendo la luminosidad para diferentes corrimientos al rojo, como en \cite{cite:Protheroe1}. Por otra parte, Mannheim et al. \cite{cite:Mannheim1}, calcularon el m\'aximo flujo posible proveniente de AGNs a partir de funciones de evoluci\'on de fuentes para blazares, un tipo de AGNs.
	
	Las explosiones de rayos gamma (GRBs) son flashes de fotones $\gamma$ emitidos por fuentes puntuales.
	Estas, representas unas de las explosiones mas fuertes del universo y una posible fuente de neutrinos c\'osmicos ultra energ\'eticos.
	Su emisi\'on ha sido calculada en varias condiciones:
	%
	\begin{enumerate}
	 \item En los modelos de tipo \emph{bola de fuego} o GRB fireball~\cite{cite:grb_Waxman1,cite:grb_Waxman2} la emisi\'on de fotones se produce como producto de colisiones de plasma que se mueve de manera relativista dentro de un chorros de materia (la bola de fuego). 
	\end{enumerate}
	%
	
	\begin{figure}[ht]
		\begin{center}
		\includegraphics[width=0.75\textwidth]{fig/introduccion/AGN_GRB_nufluxes}
		\caption{\label{fig:flujosAGN} Flujos predichos para AGNs y GRBs. En negro se muestran tres modelos de AGNs\cite{cite:Mannheim1,cite:BBR,cite:SteckerAGN}, mientras que en rojo dos para GRBs\cite{cite:grb_Waxman2,cite:grb_Supernova}.}
		\end{center}
	\end{figure}
	
% 	\subsection{Fuentes no convencionales}
% 	
% 	\begin{figure}[ht]
% 		\begin{center}
% 		\includegraphics[width=0.75\textwidth]{fig/introduccion/unconventional_nuFluxes}
% 		\caption{\label{fig:flujosNoConv} }
% 		\end{center}
% 	\end{figure}

% \section{Búsquedas de neutrinos cósmicos ultra energéticos}

% \section{Nacimiento de la astronom\'ia de neutrinos c\'osmicos}

