\begin{thebibliography}{1}

\bibitem{allan1971} H.R. Allan, Progress in Elementary Particles and Cosmic Ray Physics 10, 171 (1971) (North Holland Publ. Co.), and refs. therein.

\bibitem{alvarez:2013} Physical Review D, Vol. 87, No. 2. (3 Oct 2012), doi:10.1103/physrevd.87.023003 

\bibitem{jackson:1998} J.D. Jackson, Classical Electrodynamics, 3rd ed., Wiley, New York, 1998.

\bibitem{kahn:1966} F.D. Kahn, I. Lerche, Proc. R. Soc. Lond. A289 (1966) 206.

\bibitem{scholten:2008} O. Scholten, K. Werner, and F. Rusydi. A macroscopic description of coherent geomagnetic radiation from cosmic-ray air showers. Astroparticle Physics, {\bf 29}:94–103, (2008).

%\cite{Agashe:2014kda}
\bibitem{Agashe:2014kda} 
  K.~A.~Olive {\it et al.}  [Particle Data Group Collaboration],
  %``Review of Particle Physics,''
  Chin.\ Phys.\ C {\bf 38}, 090001 (2014).
  %%CITATION = CHPHD,C38,090001;%%
  %1103 citations counted in INSPIRE as of 24 May 2015

\bibitem{cite:icrc13Auger} arXiv:1307.5059 [astro-ph.HE]

\bibitem{geosintrotron} T. Huege, M. Ludwig, O. Scholten, and K.D. de Vries. The convergence of EAS radio emission models and a detailed comparison of REAS3 and MGMR simulations. Nuclear Instruments and Methods in Physics Research Section A: Accelerators, Spectrometers, Detectors and Associated Equipment, 662(Supplement 1):S179–S186, 2012. ARENA 2010.

\bibitem{atmosphericField} S. Buitink et al. - LOPES Collaboration. Amplified radio emission from cosmic ray air showers in thunderstorms. Astronomy and Astrophysics, 467:385–394, 2007.
 

\bibitem{bremsstrahlungMolec} P. W. Gorham et al. - WMAP. Observations of microwave continuum emission from air shower plasmas. Physical Review D, 78(3):032007, 2008.

 
\bibitem{cite:hauge_rec} arXiv:1408.2346 [astro-ph.IM]
\bibitem{cite:lofar_rec} S. Buitink et al. PRD 90, 082003 (2014)

\bibitem{cite:grand_prop} https://indico.in2p3.fr/event/10976/material/paper/1.pdf
\bibitem{cite:grand_tec} https://indico.in2p3.fr/event/10976/material/paper/0.pdf

\bibitem{cite:nu_reviews}
F.~Halzen {\it et al.}, Rep. Prog. Phys. {\bf 65}, 1025 (2002);\newline
P.~Bhattacharjee {\it et al.}, Phys. Rep. {\bf 327}, 109 (2000);\newline
J.K. Becker, Phys. Rep. {\bf 458}, 173 (2008). 
\bibitem{cite:Greisen} K. Greisen, Phys. Rev. Lett. \textbf{16}, 748 (1966).
\bibitem{cite:Zatsepin} G.T. Zatsepin and V.A. Kuzmin, JETP Lett. \textbf{4}, 78 (1966).
% \bibitem{cite:Learned} J. G. Learned and K. Mannheim, Annual Review of Nuclear and Particle Science \textbf{50}, 679 (2000).
\bibitem{cite:Hires} HiRes Collaboration, Phys.Rev.Lett. \textbf{100}, 101101 (2008).
\bibitem{cite:Agasa}  M. Takeda {\it et al.}, Astropart.Phys. \textbf{19}, 447 (2003).
\bibitem{cite:AugerSpectrum} Pierre Auger Collaboration, Eur. Phys. J. Plus, \textbf{127}, 87 (2012).
\bibitem{cite:photonInt1} R. J. Gould and G. Schreder, Phys. Rev. Lett., \textbf{16}, 252 (1966).
\bibitem{cite:photonInt2}  J. V. Jelley, Phys. Rev. Lett. \textbf{16}, 479 (1966).
\bibitem{cite:IR} G. Setti {\it et al.}, Nature Physical Science \textbf{231}, 57 (1971).
\bibitem{cite:URB} R. J. Protheroe and P. L. Biermann, Astropart. Phys. \textbf{6}, 45 (1996).\\
% [Erratum-ibid.]
% [Erratum\-ibid. \textbf{7}, 181 (1997)]
Erratum-ibid. \textbf{7}, 181 (1997). astro-ph/9605119.
\bibitem{cite:fotones_camino} M. Risse {\it et al.}, Mod. Phys. Lett. A \textbf{22}, 749 (2007).
\bibitem{cite:GZK1} V.S. Berezinsky and G. T. Zatsepin, Phys. Lett. {\bf 28B}, 423 (1969).\newline
V.S. Berezinsky y G. T. Zatsepin, Sov. J. Nucl. Phys. {\bf 11}, 111 (1970).
\bibitem{cite:GZK2} F. W. Stecker, Astrophys. Space Sci. {\bf 20}, 47 (1973).\newline
F. W. Stecker, ApJ {\bf 228}, 919 (1979).
\bibitem{cite:nuAGN} F. W. Stecker  {\it et al.}, Physical Review Letters \textbf{66}, 2697 (1991).
\bibitem{cite:nuGRB} E. Waxman and J. Bahcall,  Physical Review Letters \textbf{78}, 2292 (1997).
\bibitem{cite:nuTopDefects} S. Yoshida {\it et al.}, Astrophys.J. \textbf{479}, 547 (1997).
\bibitem{cite:nuZBurst_init} T. Weiler,  Phys. Rev. Lett. \textbf{49}, 234 (1982).
% \bibitem{cite:nuZBurst}  O. E. Kalashev {\it et al.}, Phys. Rev. D \textbf{66}, 063004 (2002).
\bibitem{cite:nuConstraintsTD} L. A. Anchordoqui and T. Montaruli, Annual Review of Nuclear and Particle Science \textbf{60}, 129 (2010).
\bibitem{cite:nuEngel} R. Engel {\it et al.}, Phys. Rev. D \textbf{64}, 093010 (2001).
\bibitem{cite:nuAve} M. Ave {\it et al.}, Astroparticle Physics \textbf{23}, 19 (2005).
\bibitem{cite:nuAhlers1} M. Ahlers {\it et al.}, Phys. Rev. D \textbf{72},  023001 (2005).
\bibitem{cite:nuAllard1}  D. Allard {\it et al.}, Journal of Cosmology and Astroparticle Physics \textbf{09},  005 (2006).
\bibitem{cite:nuYuksel}H. Yuksel and M. D. Kistler, Phys. Rev. D \textbf{75}, 083004 (2007).
\bibitem{cite:nuHooper} D. Hooper {\it et al.}, Astroparticle Physics \textbf{23}, 11 (2005).
\bibitem{cite:nuFeComposition} F. W. Stecker, Phys. Rev. \textbf{180} no. 5, 1264 (Apr, 1969).
\bibitem{cite:augerComposition} Pierre Auger Collaboration, Physical Review Letters \textbf{104} no. 9, 091101 (2010).
\bibitem{cite:taComposition} Telescope Array Collaboration, 
``Ultra-high energy cosmic-ray spectra measured by the Telescope Array experiment from hybrid observations", 
32nd International Cosmic Ray Conference. 2011.
\bibitem{cite:Lambda} N. A. Bahcall, J. P. Ostriker {\it et al.}, Science \textbf{284}, 1481 (1999).
\bibitem{cite:LambdaM} S. M. Carroll, Living Reviews in Relativity \textbf{4}, 1 (2001).
\bibitem{cite:SFR} A. J. Bunker {\it et al.}, Mon.Not.Roy.Astron.Soc. \textbf{355}, 374 (2004).
\bibitem{cite:SFR_citedByAllard1} D. Hopkins and J. F. Beacom, ApJ \textbf{651}, 142 (2006).\\
H. Yuksel {\it et al.}, ApJ \textbf{683}, L5 (2008).
\bibitem{cite:Lemoine_Waxman} M. Lemoine and E. Waxman, Journal of Cosmology and Astro-Particle Physics \textbf{11}, 9 (2009).
\bibitem{cite:FRII_Wall} J. V. Wall {\it et al.}, Astronomy and Astrophysics \textbf{434}, 133 (2005).
\bibitem{cite:Le_Dermer} T. Le and C. D. Dermer, ApJ \textbf{661}, 394 (2007).
\bibitem{cite:EstebanSilivia} E. Roulet {\it et al.}, JCAP01, 028 (2013).
%AGN
\bibitem{cite:GHZ}  T. K. Gaisser, F. Halzen, and T. Stanev, Physics Reports \textbf{258} no. 3, 173 (1995).
\bibitem{cite:Fermi1} E. Fermi, Phys. Rev. \textbf{75} (1949).
\bibitem{cite:Protheroe1} R. J. Protheroe, “High energy neutrinos from blazars,” arXiv:astro-ph/9607165.
\bibitem{cite:Mannheim1}  K. Mannheim, R. J. Protheroe, and J. P. Rachen,  Phys. Rev. D \textbf{63} no. 2, 023003 (2001).
\bibitem{cite:BBR}  J. K. Becker {\it et al.}, Astropart. Phys. \textbf{23} 355, (2005).
\bibitem{cite:SteckerAGN} F. W. Stecker, Phys. Rev. D \textbf{72} no. 10,  107301 (2005).
%GRB
\bibitem{cite:grb_Waxman1} E. Waxman, S. R. Kulkarni, and D. A. Frail,  Astrophys.J. \textbf{497}, 288 (1998).
\bibitem{cite:grb_Waxman2} E. Waxman and J. N. Bahcall, Astrophys.J. \textbf{541}, 707 (2000).
\bibitem{cite:grb_Supernova} S. Razzaque, P. Meszaros, and E. Waxman, Physical Review Letters \textbf{90} no. 24, 241103 (2003).
\bibitem{cite:Kalashev} O. Kalashev {\it et al.}, Phys. Rev. D \textbf{66}, 063004 (2002).
%WB
\bibitem{cite:WaxmanBahcall1} E. Waxman and J. N. Bahcall, Phys.Rev. \textbf{D59},  023002 (1999).
\bibitem{cite:Waxman1}  E. Waxman,  Astrophys.J. \textbf{452},  L1 (1995).
\bibitem{cite:nuSearchReview1} F. Halzen and S. R. Klein, Phys.Today \textbf{61N5}, 29 (2008).
%LakeBaikal1
\bibitem{cite:LakeBaikal1} BAIKAL and TUNKA Collaboration, Nuclear Instruments and Methods A \textbf{442} no. 1, 368 (2000).
%IceCube
\bibitem{cite:IceCube1} IceCube Collaboration, Phys. Rev. D \textbf{83} no. 9,  092003 (2011).
\bibitem{cite:IceCubePev} IceCube Collaboration, Phys. Rev. Lett. \textbf{111}, 021103 (2013).
%Anteares
\bibitem{cite:Antares1} Antares Collaboration, Astroparticle Physics \textbf{26}, 314 (2006).
%Nemo
\bibitem{cite:Nemo1} NEMO Collaboration, Nuclear Physics B Proceedings Supplements \textbf{143},  359–362 (2005).
%Nestor
\bibitem{cite:Nestor1}  The Nestor Collaboration,  Nuclear Instruments and Methods in Physics Research A \textbf{552} 420 (2005).
%Km3Net
\bibitem{cite:km3Net1}  Km3NeT Collaboration, Nuclear Instruments and Methods in Physics Research A \textbf{602}, 98 (2009).
%Askaryan
\bibitem{cite:Askaryan}  G. A. Askar’yan, Soviet Journal of Experimental and Theoretical Physics \textbf{21}, 658 (1965).
\bibitem{cite:RICE} RICE Collaboration, Phys.Rev. \textbf{D73}  082002 (2006).
\bibitem{cite:Anita1} ANITA Collaboration, Phys.Rev.Lett. \textbf{103},  051103 (2009).
\bibitem{cite:Anita2}ANITA Collaboration,Phys. Rev. D \textbf{82}, 022004 (2010);\\
Erratum arXiv:1011.5004v1 [astro-ph]
\bibitem{cite:Glue}  P. W. Gorham {\it et al.},  Phys. Rev. Lett. \textbf{93}  041101 (2004).
\bibitem{cite:Kalazin} R. Dagkesamansky {\it et al.}, Int.J.Mod.Phys. \textbf{A21S1},  142 (2006).
\bibitem{cite:Parkes} C. W. James {\it et al.}, Mon.Not.Roy.Astron.Soc. \textbf{379}, 1037 (2007).
\bibitem{cite:NuMoon}  O. Scholten {\it et al.}, J.Phys.Conf.Ser. \textbf{81}  012004 (2007).
\bibitem{cite:SKA} www.skatelescope.org
\bibitem{cite:LUNASKA}  C. W. James {\it et al.}, Phys.Rev. \textbf{D81}  042003 (2010).
\bibitem{cite:ARA} P. Allison {\it et al.}, arXiv:1105.2854 [astro-ph.IM].
\bibitem{cite:ARIANNA} S. W. Barwick, Journal of Physics Conference Series \textbf{60}, 276 (2007).
\bibitem{cite:ARIANNA2}  T. Barrella, S. Barwick, and D. Saltzberg, Journal of Glaciology \textbf{57}, 61–66 (2011).
\bibitem{cite:SalSa1}  P. Gorham {\it et al.}, Nuclear Instruments and Methods A, \textbf{490} no. 3,  476 (2002).
\bibitem{cite:SalSa2} J. Alvarez-Muniz {\it et al.}, Int.J.Mod.Phys. \textbf{A21S1}, 55 (2006).
\bibitem{cite:SalSa3} R. D. Stewart and R. R. Unterberger, Geophysics \textbf{41} no. 1, 123 (1976).
\bibitem{cite:HiresNu} HiRes Collaboration, arXiv:0803.0554 [astro-ph].
\bibitem{cite:HiresNu2} HiRes Collaboration, arXiv:0707.4417 [astro-ph].
\bibitem{cite:AugerNIM1} Pierre Auger Collaboration, Nucl.Instrum.Meth. \textbf{A523},  50 (2004).
\bibitem{cite:AugerNu3} Pierre Auger Collaboration, Phys. Rev. D \textbf{84}, 122005 (2011).
\bibitem{cite:AugerNu1} Pierre Auger Collaboration, Phys. Rev. Lett., \textbf{100}, 211101 (2008).
% \bibitem{cite:AugerNu2} Pierre Auger Collaboration, Phys. Rev. D, \textbf{79}, 102001 (2009).
\bibitem{cite:AugerNu4} Pierre Auger Collaboration, Astrophysical Journal Letters, \textbf{755}, L4 (2012).

%%%%%%%%%%%%%%%%%%%%%%%%%%%%%%%%%%%%%%%%%%%%%%%%%%%%%%%%%%%%%%%%%
%%%%%%%%%%%%%%%%%%%%%%%%%%%%%%%%%%%%%%%%%%%%%%%%%%%%%%%%%%%%%%%%%
%%%%%%%%%%%%%%%%%%%%%%%%%%%%%%%%%%%%%%%%%%%%%%%%%%%%%%%%%%%%%%%%%
% Atmospheric Showers
\bibitem{cite:Auger_photon_limit}
J. Abraham {\it et al.} Pierre Auger Collaboration, Astropart. Phys. \textbf{27}, 155 (2007);
J. Abraham {\it et al.} Pierre Auger Collaboration, Astropart. Phys. {\bf 31}, 399-406 (2009).
\bibitem{hei54} W. Heitler, The Quantum Theory of Radiation, third ed., Oxford University Press, London, p. 386 (Section38), (1954).
\bibitem{cite:ines} I. Vali\~no (PhD thesis), Univ. de Santiago de Compostela, ISBN: 9788497509664 (2008).
\bibitem{cite:tesisJavier} J. Tiffenberg (PhD thesis), Univ. de Buenos Aires (2011).
\bibitem{cite:Gandhi} R. Gandhi {\it et al.}, Astropart. Phys. 5 (1996) 81.
\bibitem{cite:nuOscCosmo} J. G. Learned {\it et al.}, Astropart. Phys. \textbf{3}, 267 (1995).
\bibitem{cite:nuOscDiscovery} Super-Kamiokande Collaboration, Phys. Rev. Lett. \textbf{81}, 1562 (1998). 
\bibitem{cite:esFargion} D. Fargion, Astrophys. J. \textbf{570}, 909 (2002).
\bibitem{cite:esLetessier} A. Letessier-Selvon, AIP Conf. Proc. \textbf{566}, 157 (2001).
\bibitem{cite:esFeng} J. L. Feng {\it et al.}, Phys. Rev. Lett. \textbf{88}, 161102 (2002).
\bibitem{cite:TA} M. Fukushima, Prog. Theor. Phys. Supplement \textbf{151} 206 (2003). 
\bibitem{cite:yakutsk} G. I. Rubtsov {\it et al.}, Phys. Rev. D \textbf{73}, 063009 (2006)
\bibitem{cite:agasa} AGASA Collaboration, NIM \textbf{311}, 338 (1992).
\bibitem{cite:HPark} R. N. Coy {\it et al.}, Astroparticle Physics, 6, (1997).

%%%%%%%%%%%%%%%%%%%%%%%%%%%%%%%%%%%%%%%%%%%%%%%%%%%%%%%%%%%%%%%%%
%%%%%%%%%%%%%%%%%%%%%%%%%%%%%%%%%%%%%%%%%%%%%%%%%%%%%%%%%%%%%%%%%
%%%%%%%%%%%%%%%%%%%%%%%%%%%%%%%%%%%%%%%%%%%%%%%%%%%%%%%%%%%%%%%%%
% Observatory
\bibitem{cite:nim} Pierre Auger Collaboration, Nucl. Instrum. Meth. A \textbf{523}, 50 (2004).
\bibitem{designReport} Pierre Auger Project Design Report, (1997).
\bibitem{auger_prop04} Pierre Auger Colaboration, Nuclear Instruments \& Methods A, \textbf{523}, 50 (2004).
\bibitem{cite:calibration}  X. Bertou et al., Nuclear Instruments \& Methods A \textbf{568}, 839 (2006).
\bibitem{cite:trigger} Pierre Auger Collaboration, Nuclear Instruments and Methods A \textbf{613}, 29 (2010).
%%%%%%%%%%%%%%%%%%%%%%%%%%%%%%%%%%%%%%%%%%%%%%%%%%%%%%%%%%%%%%%%%
%%%%%%%%%%%%%%%%%%%%%%%%%%%%%%%%%%%%%%%%%%%%%%%%%%%%%%%%%%%%%%%%%
%%%%%%%%%%%%%%%%%%%%%%%%%%%%%%%%%%%%%%%%%%%%%%%%%%%%%%%%%%%%%%%%%
% Nu Simulations
\bibitem{cite:cooper_sarkar} A.~Cooper-Sarkar and S.~Sarkar, JHEP {\bf 0801}, 075 (2008).
\bibitem{cite:cooper_sarkar2}Amanda Cooper-Sarkar, Philipp Mertsch, Subir Sarkar. JHEP \textbf{08}, 042 (2011).
\bibitem{cite:Armesto2008} N. Armesto {\it et al.}, Phys. Rev. D \textbf{77}, 013001 (2008).
\bibitem{cite:ASW} N. Armesto {\it et al.}, Phys. Rev. Lett. \textbf{94}, 022002 (2005).
% \bibitem{cite:ASW} N. Armesto {\it et al.}, Phys. Rev. Lett., \textbf{94}, 022002 (2005).
\bibitem{cite:Dutta1} S. I. Dutta {\it et al.}, Phys. Rev. D \textbf{63}, 094020 (2001).
\bibitem{cite:PetrukhinShestakov} A. A. Petrukhin et V.V. Shestakov, Can. J. Phys. \textbf{46}, S377 (1968).
\bibitem{cite:KokoulinPetrukhin} R. P. Kokoulin et A. A. Petrukhin, Proc. of the $12^{th}$ ICRC, Australia, Vol \textbf{6}, 2436 (1971).
% \bibitem{cite:ALLM97} H. Abramowicz {\it et al.}, Phys. Lett. B \textbf{269}, 465  (1991);\\ 
% H. Abramowicz and A. Levy, hep-ph/9712415.
\bibitem{cite:ALLM} H. Abramowicz {\it et al.}, Phys. Lett. B \textbf{269}, 465  (1991);\\ 
H. Abramowicz and A. Levy, hep-ph/9712415.
% A. A. Kochanov {\it et al.}, Procs. $29^{\rm th}$ International Cosmic Ray Conference 2005, $\#$0300, Pune, India.
% Presented at the 29th ICRC, Aug 3-10 2005, Pune.
\bibitem{cite:PT} D. A. Timashkov and A. A. Petrukhin, Procs. $29^{\rm th}$ International Cosmic Ray Conference, Vol 9, 89, 2005

\bibitem{cite:TAUOLA} S. Jadach {\it et al.}, Comput. Phys. Commun. {\bf 76}, 361 (1993).
\bibitem{cite:corsika} \emph{http://www-ik.fzk.de/corsika}
\bibitem{cite:aires} S. Sciutto, AIRES. \emph{http://www.fisica.unlp.edu.ar/auger/aires}
\bibitem{cite:SIBYLL} \emph{http://arxiv.org/abs/0906.4113}
\bibitem{cite:QGSJET} \emph{http://arxiv.org/abs/0706.3784}
\bibitem{cite:QGSJETII} \emph{http://arxiv.org/abs/astro-ph/0412591v1} and \emph{http://arxiv.org/abs/hep-ph/0412332}
\bibitem{cite:thinning} A. M. Hillas, Proceedings of the $17^{\rm th}$ International Cosmic Ray Conference, Paris Vol. 8, 193 (1981).
\bibitem{cite:offline} S. Argiro {\it et al.}, Nucl. Instrum. Meth. A \textbf{580}, 1485-1496 (2007).
\bibitem{cite:stationSim} Pierre Auger Collaboration, Procs. $30^{\rm th}$ International Cosmic Ray Conference 2007, $\#$0300, Merida, Mexico.
\bibitem{cite:Billoir_unthinning} P.~Billoir, Astropart. Phys. {\bf 30}, 270-285 (2008).
\bibitem{cite:GEANT4} \emph{http://geant4.web.cern.ch/geant4/}
%%%%%%%%%%%%%%%%%%%%%%%%%%%%%%%%%%%%%%%%%%%%%%%%%%%%%%%%%%%%%%%%%
%%%%%%%%%%%%%%%%%%%%%%%%%%%%%%%%%%%%%%%%%%%%%%%%%%%%%%%%%%%%%%%%%
%%%%%%%%%%%%%%%%%%%%%%%%%%%%%%%%%%%%%%%%%%%%%%%%%%%%%%%%%%%%%%%%%
% Reconstruction and Inclined selection
\bibitem{cite:Auger_trigger} Pierre Auger Collaboration, Nucl. Instr. and Meth. A {\bf 613}, 29-39 (2010).
\bibitem{cite:pmtCuts} I. Lhenry-Yvon and P. L. Ghia,``Implementation of PMTs Quality Cuts in Auger Sd Data'', Auger internal note GAP-2009-080 (2009);
I. Lhenry-Yvon, ``CDAS production v1r0: overview of new data avialable'', Auger internal note GAP-2012-077 (2012).
\bibitem{cite:atmo_mu} P. Lipari, Astropart. Phys. \textbf{1}, 195–227 (1993).
\bibitem{cite:atmo_muGAP} J. Blumer, ``On the influence of accidental muons on air shower detection'', 
Auger internal note GAP-2008-110 (2008).
\bibitem{cite:traceCleaning}  P. Billoir, ``FADC trace cleaning in Surface Detector through a segmentation procedure'', 
Auger internal note GAP-2005-074 (2005).
\bibitem{cite:TopDownSel}  P. Billoir, ``Top-down Selection of Events and Stations in Surface Detector Triggers'', 
Auger internal note GAP-2006-072 (2006).
\bibitem{cite:ShowerFront} P. Billoir, ``Reconstruction of first year EA events from the Surface Detector'', 
Auger internal note GAP-2002-044 (2002).
%%%%%%%%%%%%%%%%%%%%%%%%%%%%%%%%%%%%%%%%%%%%%%%%%%%%%%%%%%%%%%%%%
%%%%%%%%%%%%%%%%%%%%%%%%%%%%%%%%%%%%%%%%%%%%%%%%%%%%%%%%%%%%%%%%%
%%%%%%%%%%%%%%%%%%%%%%%%%%%%%%%%%%%%%%%%%%%%%%%%%%%%%%%%%%%%%%%%%
% Identification
\bibitem{cite:blind} A. Roodman, PHYSTAT2003, 166 (2003).
%%%%%%%%%%%%%%%%%%%%%%%%%%%%%%%%%%%%%%%%%%%%%%%%%%%%%%%%%%%%%%%%%
%%%%%%%%%%%%%%%%%%%%%%%%%%%%%%%%%%%%%%%%%%%%%%%%%%%%%%%%%%%%%%%%%
%%%%%%%%%%%%%%%%%%%%%%%%%%%%%%%%%%%%%%%%%%%%%%%%%%%%%%%%%%%%%%%%%
% Exposure
\bibitem{cite:BertouNuTau} X. Berotu {\it et al.}, Astropart.Phys, \textbf{17}  183(2002).
% \bibitem{cite:ALLM} A. A. Kochanov {\it et al.}, Procs. $29^{\rm th}$ International Cosmic Ray Conference 2005, $\#$0300, Pune, India.
% Presented at the 29th ICRC, Aug 3-10 2005, Pune
% 
% \bibitem{cite:CKMT}A. Capella {\it et al.}, Phys. Lett. B, \textbf{337}, 358 (1994).
%%%%%%%%%%%%%%%%%%%%%%%%%%%%%%%%%%%%%%%%%%%%%%%%%%%%%%%%%%%%%%%%%
%%%%%%%%%%%%%%%%%%%%%%%%%%%%%%%%%%%%%%%%%%%%%%%%%%%%%%%%%%%%%%%%%
%%%%%%%%%%%%%%%%%%%%%%%%%%%%%%%%%%%%%%%%%%%%%%%%%%%%%%%%%%%%%%%%%
% Results
\bibitem{cite:DecUncalibrated} H. Wahlberg, ``Long Term Evolution of the VEM Area to Peak ratio'', 
Auger internal note GAP-2002-044 (2002).
% \bibitem{cite:Neyman} J. Neyman, Philos. Trans. R. Soc. London \textbf{A236}, 333 (1937).
\bibitem{cite:Feldman-Cousins} G. J. Feldman and R.D Cousins, Phys. Rev. D \textbf{57}, 3873 (1998).
% \bibitem{cite:cinturon} W. T. Eadie {\it et al.}, Statistical Methods in Experimental Physics, North-Holland, Amsterdam, (1971).
\bibitem{cite:Conrad_limit} J. Conrad {\it et al.}, Phys. Rev. D \textbf{67} 12002 (2003).
\bibitem{cite:Abbasi11b} IceCube Collaboration, Astrophys. J. \textbf{732}, 18 (2011).
\bibitem{cite:Adrian11} Antares Collaboration, Astrophys. J. Lett. \textbf{743}, L14 (2011).
\bibitem{cite:Cuoco08} A. Cuoco and S. Hannestad, Phys. Rev. D  \textbf{78}, 023007 (2008).
\bibitem{cite:Kachel09} M. Kachelriess, S. Ostapchenko and R. Tomas, New J. Phys. \textbf{11}, 065017 (2009).
\bibitem{cite:LUNASKA11} LUNASKA Collaboration, MNRAS \textbf{410}, 885 (2011).


% \bibitem{cite:lluviasDeAltaEnergia} [Pierre Auger Collaboration], Phys. Lett. B \textbf{685}, 239-246 (2010).
% \bibitem{cite:coc22} G. Cocconi, Nuovo Cimento \textbf{3}, 1422  (1956).
% \bibitem{cite:pelo_largo} M. Longair, High Energy Astrophysics, Vol. 1, Cambridge University Press, Cambridge (1992).
% \bibitem{cite:nubes_mag} M. de Avillez, Astrophysics and Space Science \textbf{292}, 207-214 (2004).
% \bibitem{cite:fermi1_100TeV}  P.O. Lagage and C.J. Cesarsky, A\&A \textbf{118}, 223 (1983).
% \bibitem{cite:jets} F. M. Rieger, Proc. JEM-EUSO Workshop (2008).
% \bibitem{cite:drury} L O'C Drury, Rep. Prog. Phys. \textbf{46}, 973-1027 (1983).
% \bibitem{cite:cmb_pyw} A.A. Penzias y R.W. Wilson, Astrophysical Journal \textbf{142}, 419–421 (1965).
% \bibitem{cite:cmb_wmap} D.N. Spergel {\it et al.}, Astrophysical Journal \textbf{148}, 175–194 (2003). 
% \bibitem{cite:bigbang} R.A. Alpher, Nature \textbf{162}, 774 (1948).  

% \bibitem{cite:Cronin} J. W. Cronin, Nuclear Physics, 29B, (1992).


% \bibitem{cite:sno} Q.R. Ahmad {\it et al.}, Phys. Rev. Lett. \textbf{87}, 071301 (2001).
% \bibitem{cite:sk} Y.Fukudaa {\it et al.} [Super-Kamiokande Collaboration], Phys. Rev. Lett. \textbf{81}, 1562-1567 (1998).
% \bibitem{cite:Wax_Bah} E. Waxman y J. Bahcall, Phys. Rev. D \textbf{59} 023002-1 (1999).
% \bibitem{cite:IceCube_des} P. A. Toale {\it et al.} [IceCube Collaboration], Proc. 27th Inter. Cosmic Ray Conf., Hamburg, Germany, (2001).
% \bibitem{cite:RICE_des} I. Kravchenko {\it et al.} [RICE Collaboration], Phys.Rev. D \textbf{73} 082002 (2006).
% \bibitem{cite:glue} Peter Gorham {\it et al.}, Phys. Rev. E \textbf{62}, 8590 (2000).
% \bibitem{cite:forte} N. G. Lehtinen {\it et al.}, Particle Astrophysics Instrumentation, Proc. SPIE \textbf{4858}, 296-304 (2003).
% \bibitem{cite:ANITA_des} J. T. Link {\it et al.} [ANITA Collaboration], Proc. 29th Inter. Cosmic Ray Conf., Pune, India, (2005).
% \bibitem{cite:baikal} V. Aynutdinov et al. [BAIKAL Collaboration], Astropart. Phys. \textbf{25}, 140 (2006).
% \bibitem{cite:amanda} R. Wischnewski {\it et al.} [AMANDA Collaboration] Nucl. Phys. B \textbf{11}, 0510 (2002).
% \bibitem{cite:ANTARES} A. Kouchner {\it et al.} [ANTARES Collaboration], Proc. 30th Inter. Cosmic Ray Conf., Mérida, México (2007).
% \bibitem{cite:NESTOR} G. Aggouras {\it et al.} [NESTOR Collaboration], Nucl. Instrum. Meth. A \textbf{567}, 452 (2006).
% \bibitem{cite:PDG} C. Amsler {\it et al.} [Particle Data Group], Physics Letters B {\bf 667}, 1 (2009).
% \bibitem{cite:nuOsc} H. Athar {\it et al.}, Phys. Rev. D \textbf{62}, 103007 (2000).



% \bibitem{cite:upgoing} [Pierre Auger Collaboration], Phys. Rev. D \textbf{79}, 102001 (2009).




% \bibitem{cite:herwig} {\sc herwig 6.5}, G. Corcella {\it et al.}, JHEP \textbf{0101}, 010 (2001). hep-ph/0210213.


% \bibitem{cite:zas_tau} E. Zas, New Jour. Phys. \textbf{7}, 130 (2005).
% \bibitem{cite:propagaTierra} O. Blanch-Bigas {\it et al.}, Phys. Rev. D \textbf{77}, 103004 (2008)



% \bibitem{cite:Fisher} R. Fisher, Ann of Eugenics {\bf7}, 179 (1936).
% \bibitem{cite:Roe_03} S.Roe, PHYSTAT2003, 215-217 (2003).
% \bibitem{cite:fullMixing} J. G. Learned {\it et al.}, Astropart. Phys. \textbf{3}, 267 (1995). H. Athar {\it et al.}, Phys. Rev. D \textbf{62}, 103007 (2000).
% \bibitem{cite:cteq} \emph{http://zebu.uoregon.edu/~parton/partonCTEQ.html}
% \bibitem{cite:PYTHIA} \emph{http://home.thep.lu.se/~torbjorn/Pythia.html}
% \bibitem{cite:Herwig++} \emph{http://projects.hepforge.org/herwig/}
% \bibitem{cite:cteq6} \emph{http://hep.pa.msu.edu/cteq/public/cteq6.html}
% \bibitem{cite:HERAPDF1.0} F.D. Aaron {\it et al.} [H1 and ZEUS], JHEP01 109, 1 (2010).
% \bibitem{cite:Auger_spectrum}
% J. Abraham {\it et al.} [Pierre Auger Collaboration], Phys. Rev. Lett. \textbf{101}, 061101 (2008).
% R.U. Abbasi {\it et al.} [HiRes], Phys. Rev. Lett. \textbf{100}, 101101 (2008).
% \bibitem{cite:xs_muon_pp} R. P. Kokoulin {\sl et al.}, in Procs. XII International Conference on Cosmic Rays, Hobart, Tasmania, Australia (1971).
% \bibitem{cite:xs_muon_br} A. A. Petrukhin {\sl et al.}, Can. J. Phys. \textbf{46}, S377 (1968).
% \bibitem{cite:xs_muon_dis_bb} L. B. Bezrukov {\sl et al.}, Yad. Fiz. \textbf{33}, 1195 (1981) y Sov. J. Nucl. Phys. \textbf{33}, 635 (1981).
% \bibitem{cite:prompt_muon_1} R. Enberg {\sl et al.}, Phys. Rev. D \textbf{78}, 043005 (2008).
% \bibitem{cite:prompt_muon_2} M. Thunman {\sl et al.}, Astropart.Phys. \textbf{5}, 309-332 (1996).




% \bibitem{cite:Tiffenberg_icrc09} J.~Tiffenberg [Pierre Auger Collaboration], Procs. $31^{\rm st}$ ICRC 2009, 0180, Lodz, Polonia (2009).
% \bibitem{cite:nu_limits}
% M. Ackermann {\it et al.} [AMANDA], Astrophys. J. 675, 1014-1024 (2008);
% %Keiichi Mase {\it et al.} [IceCube], Procs. $31^{\rm st}$ International Cosmic Ray Conference 2009, $\#$0861, Lodz, Poland;
% R. Abbasi {\it et al.} [IceCube], Phys. Rev. D in press, arXiv:1009.1442; Private communication, IceCube Collaboration, (2010);
% I. Kravchenko {\it et al.} [RICE], Phys. Rev. D {\bf 73}, 082002 (2006);
% P.W. Gorham {\it et al.} [ANITA], arXiv:0812.2715v1;
% R. Abbasi {\it et al.} [HiRes], Astrophys. J. 684, 790-793 (2008);
% K. Martens [HiRes], arXiv:0707.4417.

% \bibitem{cite:nu_GZK_Ahlers} M. Ahlers {\it et al.}, Astropart. Phys. \textbf{34}, 106 (2010).
% \bibitem{cite:nu_GZK_Kotera} K. Kotera {\it et al.}, JCAP \textbf{10}, 013 (2010).


% \bibitem{cite:augerScience} [Pierre Auger Collaboration], Science \textbf{318}, 5852, 938-943 (2007)



% %_________________________________________________________________________________________
% \bibitem{coc22} G. Cocconi, Nuovo Cimento, 3, 1422
% \bibitem{gre66} K. Greisen, Phys. Rev. Lett. 16, 748 (1966).
% \bibitem{zat66} G.T. Zatsepin \& V.A. Kuzmin, JETP Lett. 4 78 (1966).
% \bibitem{hei54} W. Heitler, The Quantum Theory of Radiation, third ed., Oxford University Press, London, p. 386 (Section38), (1954).
% \bibitem{icrctri} D. Allard y otros, ICRC, (2005).
% \bibitem{icrcLDF} P. Bauleo y otros, ICRC, (2005).
% \bibitem{gap-05-59} M. Gomes Berizo y otros, GAP Note 59, (2005).
% \bibitem{gap-05-05} P. L. Ghia, GAP Note 5, (2005).
% \bibitem{auger_prop04} Auger Colaboration, Nuclear Instruments \& Methods in Physics Research, A523, 50-95, (2004).
% \bibitem{lateralMuon} P. Billoir, GAP Note 86, (2005).
% \bibitem{ROOT} http://root.cern.ch
% 
% \bibitem{Aharonian} F. A. Aharonian y otros, Physics Review D, 50, (1994).
% \bibitem{satrec} I. C. Maris y otros, GAP Note 12, (2006).
% \bibitem{designReport} Pierre Auger Project Design Report, (1997).
% \bibitem{gap-05-47} P. Bauleo y otros, GAP Note 47, (2004).
% \bibitem{pribateTri} P. Bauleo y otros, comunicaci\'on privada, (2005).
% \bibitem{acceleration} P. Bhattacharjee, Physics Reports, agosto, (1999).
% \bibitem{HERA} HERA \& COMPAS, CERN-HERA 87-01, (1987).
% 
% \bibitem{NKG} K. Greisen, Progress in Cosmic Ray Physics, 3, (1956).
% \bibitem{aceleracion1} P. Bhattacharjee y otros, Physics Reports, (1999).
% \bibitem{greisenNKG} K. Greisen, Annual Review of Nuclear Science, 10, 63, (1960).
% \bibitem{coy81} R. N. Coy y otros, ICRC, (1981).
% 
% \bibitem{cite:Cronin} J. W. Cronin, Nuclear Physics, 29B, (1992).
% 
% \bibitem{cite:herwig} {\sc herwig 6.5}, G. Corcella {\it et al.}, JHEP \textbf{0101}, 010 (2001). hep-ph/0210213.
% 
% \bibitem{cite:nim} J. Abraham et al. [Pierre Auger Collaboration], Nucl. Instrum. Meth. A \textbf{523}, 50 (2004).
% 
% \bibitem{cite:Auger_trigger} J. Abraham {\it et al.} [Pierre Auger Collaboration], Nucl. Instr. and Meth. A {\bf 613}, 29-39 (2010).
% 
% \bibitem{cite:PDG} C. Amsler {\it et al.} [Particle Data Group], Physics Letters B {\bf 667}, 1 (2009).
% 
% \bibitem{cite:atmo_mu} P. Lipari, Astropart. Phys. \textbf{1}, 195–227 (1993).
% 
% \bibitem{cite:blind} A. Roodman, PHYSTAT2003, 166-169 (2003).
% 
% \bibitem{cite:Fisher} R. Fisher, Ann of Eugenics {\bf7}, 179 (1936).
% 
% \bibitem{cite:Roe_03} S.Roe, PHYSTAT2003, 215-217 (2003).
% 
% \bibitem{cite:fullMixing} J. G. Learned {\it et al.}, Astropart. Phys. \textbf{3}, 267 (1995). H. Athar {\it et al.}, Phys. Rev. D \textbf{62}, 103007 (2000).
% 
% \bibitem{cite:cooper_sarkar} A.~Cooper-Sarkar and S.~Sarkar, JHEP {\bf 0801}, 075 (2008).
% 
% \bibitem{cite:HERWIG} G. Corcella {\it et al.}, {\sc herwig} 6.5, JHEP 0101 (2001).
% 
% \bibitem{cite:TAUOLA} S. Jadach {\it et al.}, Comput. Phys. Commun. {\bf 76}, 361 (1993).
% 
% \bibitem{cite:aires} S. Sciutto, AIRES. \emph{http://www.fisica.unlp.edu.ar/auger/aires}
% 
% \bibitem{cite:Offline} S. Argiró {\it et al.} [The Offline group - Pierre Auger Collaboration] Nucl. Instr. and Meth. A, {\bf 580}, 1485-1496 (2007).
% 
% \bibitem{cite:thinning} A. M. Hillas, Proceedings of the $17^{\rm th}$ International Cosmic Ray Conference, Paris Vol. 8, 193 (1981).
% 
% 
% \bibitem{cite:stationSim} P. L. Ghia1 [Pierre Auger Collaboration], Procs. $30^{\rm th}$ International Cosmic Ray Conference 2007, $\#$0300, Merida, Mexico.
% 
% \bibitem{cite:Billoir_unthinning} P.~Billoir, Astropart. Phys. {\bf 30}, 270-285 (2008).
% 
% \bibitem{cite:GEANT4} \emph{http://geant4.web.cern.ch/geant4/}
% 
% \bibitem{cite:cteq} \emph{http://zebu.uoregon.edu/~parton/partonCTEQ.html}
% 
% \bibitem{cite:QGSJETII} \emph{http://arxiv.org/abs/astro-ph/0412591v1} and \emph{http://arxiv.org/abs/hep-ph/0412332}
% 
% \bibitem{cite:PYTHIA} \emph{http://home.thep.lu.se/~torbjorn/Pythia.html}
% 
% \bibitem{cite:Herwig++} \emph{http://projects.hepforge.org/herwig/}
% 
% \bibitem{cite:cteq6} \emph{http://hep.pa.msu.edu/cteq/public/cteq6.html}
% 
% \bibitem{cite:corsika} \emph{http://www-ik.fzk.de/corsika}
% 
% \bibitem{cite:QGSJET} \emph{http://arxiv.org/abs/0706.3784}
% 
% \bibitem{cite:SIBYLL} \emph{http://arxiv.org/abs/0906.4113}
% 
% \bibitem{cite:HERAPDF1.0} F.D. Aaron {\it et al.} [H1 and ZEUS], JHEP01 109, 1 (2010).
% 
% \bibitem{cite:AMC-S} AMC-S, \emph{http://www-pnp.physics.ox.ac.uk/$\sim$cooper/neutrino.html}
% 
% \bibitem{cite:zas_tau} E. Zas, New Jour. Phys. \textbf{7}, 130 (2005).
% 
% \bibitem{cite:propagaTierra} O. Blanch-Bigas {\it et al.}, Phys. Rev. D \textbf{77}, 103004 (2008)
% 
% \bibitem{cite:nu_GZK_Ahlers} M. Ahlers {\it et al.}, Astropart. Phys. \textbf{34}, 106 (2010).
% 
% \bibitem{cite:nu_GZK_Kotera} K. Kotera {\it et al.}, JCAP \textbf{10}, 013 (2010).
% 
% \bibitem{cite:MPR01} K. Mannheim {\it et al.}, Phys. Rev. D  \textbf{63} 23003, (2001).
% 
% \bibitem{cite:BBR}  J. K. Becker {\it et al.}, Astropart. Phys. \textbf{23} 355, (2005).
% 
% \bibitem{cite:Sigl} O. Kalashev {\it et al.}, Phys. Rev. D \textbf{66}, 063004 (2002).
% 
% \bibitem{cite:Tiffenberg_icrc09} J.~Tiffenberg [Pierre Auger Collaboration], Procs. $31^{\rm st}$ International Cosmic Ray Conference 2009, $\#$0180, Lodz, Polonia.
% 
% \bibitem{cite:Feldman-Cousins} G. J. Feldman and R.D Cousins, Phys. Rev. D \textbf{57}, 3873 (1998).
% 
% \bibitem{cite:Neyman} J. Neyman, Philos. Trans. R. Soc. London \textbf{A236}, 333 (1937).
% 
% \bibitem{cite:cinturon} W. T. Eadie {\it et al.}, Statistical Methods in Experimental Physics, North-Holland, Amsterdam, (1971).
% 
% \bibitem{cite:Conrad_limit} J. Conrad {\it et al.}, Phys. Rev. D \textbf{67} 12002 (2003).
% 
% \bibitem{cite:nu_limits}
% M. Ackermann {\it et al.} [AMANDA], Astrophys. J. 675, 1014-1024 (2008);
% %Keiichi Mase {\it et al.} [IceCube], Procs. $31^{\rm st}$ International Cosmic Ray Conference 2009, $\#$0861, Lodz, Poland;
% R. Abbasi {\it et al.} [IceCube], Phys. Rev. D in press, arXiv:1009.1442; Private communication, IceCube Collaboration, (2010);
% I. Kravchenko {\it et al.} [RICE], Phys. Rev. D {\bf 73}, 082002 (2006);
% P.W. Gorham {\it et al.} [ANITA], arXiv:0812.2715v1;
% R. Abbasi {\it et al.} [HiRes], Astrophys. J. 684, 790-793 (2008);
% K. Martens [HiRes], arXiv:0707.4417.
% 
% \bibitem{cite:nu_reviews}
% F.~Halzen {\sl et al.} Rep. Prog. Phys. {\bf 65}, 1025 (2002);
% P.~Bhattacharjee {\sl et al.} Phys.~Rep.~{\bf 327}, 109 (2000);
% J.K. Becker, Phys. Rep. {\bf 458}, 173-246 (2008). 
% 
% 
% 
% \bibitem{cite:xs_muon_dis_bb} L. B. Bezrukov {\sl et al.}, Yad. Fiz. \textbf{33}, 1195 (1981) y Sov. J. Nucl. Phys. \textbf{33}, 635 (1981).
% 
% \bibitem{cite:xs_muon_br} A. A. Petrukhin {\sl et al.}, Can. J. Phys. \textbf{46}, S377 (1968).
% 
% \bibitem{cite:xs_muon_pp} R. P. Kokoulin {\sl et al.}, in Procs. XII International Conference on Cosmic Rays, Hobart, Tasmania, Australia (1971).
% 
% 
% \bibitem{cite:prompt_muon_1} R. Enberg {\sl et al.}, Phys. Rev. D \textbf{78}, 043005 (2008).
% \bibitem{cite:prompt_muon_2} M. Thunman {\sl et al.}, Astropart.Phys. \textbf{5}, 309-332 (1996).
% 
% \bibitem{cite:Auger_spectrum}
% J. Abraham {\it et al.} [Pierre Auger Collaboration], Phys. Rev. Lett. \textbf{101}, 061101 (2008).
% R.U. Abbasi {\it et al.} [HiRes], Phys. Rev. Lett. \textbf{100}, 101101 (2008).
% 
% \bibitem{cite:ines} I. Vali\~no (tesis de doctorado) Univ. de Santiago de Compostela, ISBN: 9788497509664 (2008).
% 
% \bibitem{cite:Auger_photon_limit}
% J. Abraham {\it et al.} [Pierre Auger Collaboration], Astropart. Phys. \textbf{27}, 155 (2007);
% J. Abraham {\it et al.} [Pierre Auger Collaboration], Astropart. Phys. {\bf 31}, 399-406 (2009).
% 
% 
% \bibitem{cite:pelo_largo} M. Longair, High Energy Astrophysics, Vol. 1, Cambridge University Press, Cambridge (1992).
% 
% \bibitem{cite:fotones_camino} M. Risse {\it et al.}, Mod. Phys. Lett. A \textbf{22}, 749-766 (2007).
% 
% \bibitem{cite:nubes_mag} M. de Avillez, Astrophysics and Space Science \textbf{292}, 207-214 (2004).
% 
% \bibitem{cite:fermi1_100TeV}  P.O. Lagage and C.J. Cesarsky, A\&A \textbf{118}, 223 (1983).
% 
% \bibitem{cite:jets} F. M. Rieger, Proc. JEM-EUSO Workshop (2008).
% 
% \bibitem{cite:coc22} G. Cocconi, Nuovo Cimento \textbf{3}, 1422  (1956).
% 
% \bibitem{cite:lluviasDeAltaEnergia} [Pierre Auger Collaboration], Phys. Lett. B \textbf{685}, 239-246 (2010).
% 
% \bibitem{cite:drury} L O'C Drury, Rep. Prog. Phys. \textbf{46}, 973-1027 (1983).
% 
% \bibitem{cite:Wax_Bah} E. Waxman y J. Bahcall, Phys. Rev. D \textbf{59} 023002-1 (1999).
% 
% \bibitem{cite:cmb_pyw} A.A. Penzias y R.W. Wilson, Astrophysical Journal \textbf{142}, 419–421 (1965).
% 
% \bibitem{cite:cmb_wmap} D.N. Spergel {\it et al.}, Astrophysical Journal \textbf{148}, 175–194 (2003). 
% 
% \bibitem{cite:bigbang} R.A. Alpher, Nature \textbf{162}, 774 (1948).  
% 
% \bibitem{cite:Greisen} K. Greisen, Phys. Rev. Lett. \textbf{16}, 748 (1966).
% 
% \bibitem{cite:Zatsepin} G.T. Zatsepin and V.A. Kuzmin, JETP Lett. \textbf{4}, 78 (1966).
% 
% \bibitem{cite:IR} G. Setti {\it et al.}, Nature Physical Science \textbf{231}, 57-59 (1971).
% 
% \bibitem{cite:URB} R. J. Protheroe y P. L. Biermann, Astropart. Phys. \textbf{6}, 45 (1996) [Erratum-ibid. \textbf{7}, 181 (1997)] astro-ph/9605119.
% 
% \bibitem{cite:nuOsc} H. Athar {\it et al.}, Phys. Rev. D \textbf{62}, 103007 (2000).
% 
% 
% \bibitem{cite:ANTARES} A. Kouchner {\it et al.} [ANTARES Collaboration], Proc. 30th Inter. Cosmic Ray Conf., Mérida, México (2007).
% \bibitem{cite:IceCube_des} P. A. Toale {\it et al.} [IceCube Collaboration], Proc. 27th Inter. Cosmic Ray Conf., Hamburg, Germany, (2001).
% \bibitem{cite:NESTOR} G. Aggouras {\it et al.} [NESTOR Collaboration], Nucl. Instrum. Meth. A \textbf{567}, 452 (2006).
% 
% \bibitem{cite:ANITA_des} J. T. Link {\it et al.} [ANITA Collaboration], Proc. 29th Inter. Cosmic Ray Conf., Pune, India, (2005).
% \bibitem{cite:RICE_des} I. Kravchenko {\it et al.} [RICE Collaboration], Phys.Rev. D \textbf{73} 082002 (2006).
% 
% 
% \bibitem{cite:TA} M. Fukushima, Prog. Theor. Phys. Supplement \textbf{151} 206-210 (2003). 
% 
% \bibitem{cite:yakutsk} G. I. Rubtsov {\it et al.}, Phys. Rev. D \textbf{73}, 063009 (2006)
% 
% \bibitem{cite:agasa} N. Chiba NIM {\it et al.} [AGASA Collaboration], NIM \textbf{311}, 338-349 (1992).
% 
% \bibitem{cite:HPark} R. N. Coy {\it et al.}, Astroparticle Physics, 6, (1997).
% 
% \bibitem{cite:baikal} V. Aynutdinov et al. [BAIKAL Collaboration], Astropart. Phys. \textbf{25}, 140 (2006).
% 
% \bibitem{cite:forte} N. G. Lehtinen {\it et al.}, Particle Astrophysics Instrumentation, Proc. SPIE \textbf{4858}, 296-304 (2003).
% 
% \bibitem{cite:amanda} R. Wischnewski {\it et al.} [AMANDA Collaboration] Nucl. Phys. B \textbf{11}, 0510 (2002).
% 
% \bibitem{cite:glue} Peter Gorham {\it et al.}, Phys. Rev. E \textbf{62}, 8590 (2000).
% 
% \bibitem{cite:sno} Q.R. Ahmad {\it et al.}, Phys. Rev. Lett. \textbf{87}, 071301 (2001).
% \bibitem{cite:sk} Y.Fukudaa {\it et al.} [Super-Kamiokande Collaboration], Phys. Rev. Lett. \textbf{81}, 1562-1567 (1998).
% 
% \bibitem{cite:upgoing} [Pierre Auger Collaboration], Phys. Rev. D \textbf{79}, 102001 (2009).
% %_________________________________________________________________________________________




% \bibitem{Auger_EAS} P. Auger y R. Maze, Compt. Rend. Acad. Sci. (Ser. II) 207 (1938) 228.
% \bibitem{Reines_Cowan} C. L. Cowen et al, Science 124, 103, 1956
% \bibitem{Desc_nu_muon} G. Dauby et al, Phys. Rev. Lett., 9, 36, 1962.
% \bibitem{Desc_tau} M. L. Perl et al, Phys. Rev. Lett., 35, 1489, 1975.
% \bibitem{Desc_nu_tau} The DoNUT collaboration, Phys. Lett., B504, 218-224, 2001.
% \bibitem{Kouchner_Antares} A. Kouchner, PhD Thesis, Abril 2001, Universidad de Paris 7, Francia.
% \bibitem{Greisen} K. Greisen, Phys. Rev. Lett. 16, 748 (1966).
% \bibitem{Zatsepin} G.T. Zatsepin and V.A. Kuzmin, JETP Lett. 4, 78 (1966).
% \bibitem{ines15} M. Ave, N.Busca, A. V. Olinto, A. A. Watson y T. Yamamoto, Proceedings of CRIS04, Catania, Italia, 2004.\newline
%  M. Ave, N.Busca, A. V. Olinto, A. A. Watson y T. Yamamoto, Proceedings of the UHECRs workshop, Leeds, Inglaterra, Julio 2004.\newline
%  M. Ave, N.Busca, A. V. Olinto, A. A. Watson y T. Yamamoto, 2005, Astropart. Phys., 23, 19.\newline
% \bibitem{Propagation_mixture} D.Allard et al., JCAP {\bf 0609} 005 (2006) 
% \bibitem{ines16} P. Bhattacharjee and G. Sigl, Phys. Rept. 327 (2000) 109.
% \bibitem{Composicion_DataParticle} W. M. Yao et al, J. Phys. G 33 (2006) 1.
% \bibitem{Espectro} M. Nagano, A. A. Watson, Rev. Mod. Phys., 72 (2000) 689.
% \bibitem{Espectro_Auger} T. Yamamoto por la colaboración de Pierre Auger, The UHECR spectrum measured at the Pierre Auger Observatory and its astrophysical implications, ICRC 2007, Mérida, México, (2007).
% \bibitem{Protones_significa_Extragalactico} D. Allard, E. Parizot, S. Goriely y A. V. Olinto, 2005, A\&A Letters, 443, 29, astro-ph/0505566.
% \bibitem{Nucleos_siginifica_Galactico} P. Blasi, R. I. Epstein y A. V. Olinto, Ap. J. Letters, 533, L123 (2000).
% \bibitem{GZKnus_pred} R. Engel et al., Phys.Rev. {\bf D64} 093010 (2001);     
% \bibitem{Correlacion_Auger} Colaboración de Pierre Auger, Astropart. Phys 29 (2008) 188.
% \bibitem{ines39} S. Yoshida, H. y. Dai, C. C. H. Jui and P. Sommers, Astrophys. J. 479 (1997) 547.
%
% \bibitem{ines58} C. Aramo, A. Insolia, A. Leonardi, G. Miele, L. Perrone, O. Pisanti and D. V. Semikoz, Astropart. Phys. 23 (2005) 65.
% \bibitem{GZK3} M. Ahlers, L. A. Anchordoqui, H. Goldberg, F. Halzen, A. Ringwald y T.J. Weiler, 2005, Phys. Rev. D 72, 023001.
% \bibitem{pesados} D. Hooper, A. Taylor, S. Sarkar, 2005, Astropart. Phys., 23, 11.
% \bibitem{Flujo_Protheroe} R. J. Protheroe, Nucl. Phys. Proc. Suppl. 77, 456, 1999.
% \bibitem{Explicacion_estim_GZK} D. Seckel y T. Stanev, PRL 95, 141101, 2005.
% \bibitem{Produccion_Neutrinos} K. Mannheim, Astropart. Phys. 3 (1995) 295. 
% \bibitem{cygnus1} M. Samorski y W. Stamm, Ap. J. Lett, 268 (1983) L17.
% \bibitem{cygnus2} J. Lloyd-Evans et al, Nature, 305 (1983) 784.
% \bibitem{Dinamo} L. Celinkier, Proc. Rencontres de Moriond, 1996.
% \bibitem{Neu_Solares_CON} C. Hettlage y K. Mannheim, Astropart. Phys. 3 (2000) 45.
% \bibitem{Neu_Solares_SIN} G. Ingelman y M. Thunman, Phys. Rev. D, 54 (1996) 4385.
% \bibitem{Menor_Valor} D. Seckel, T. Stanev y T.K. Gaisser, Ap. J.382 (1991) 652.
% \bibitem{Compatibles} D. J. Thompson et al, J. Geophys Res., 102 (1997) 14735.
% \bibitem{Manual} T. K. Gaisser, Cosmic Rays and Particle Physics, Cambridge Univ. Press, 1990.
% \bibitem{Potencia_AGN} T.K. Gaisser et al, Phys. Rep. 258 (1995) 173.
% \bibitem{AGN1} F. Piron, Observations of gamma-ray emission above 250 GeV from the blazars Markarian 501 and Markarian 421 by the CAT Cherenkov atmosferic imaging telescope, Proceeding of the XI Rencontre de Blais, astroph/9910517.
% \bibitem{AGN2} J. Holder, Observations of Mkr 421 wth the CELESTE experiment, Proceedings of Heidelberg gamma-ray symposium, astroph/0010264.
% \bibitem{AGN3} M. Punch et al (Colaboración WHIPPLE), Nature 358 (1992) 477.
% \bibitem{AGN4} D. Petry et al (Colaboración HEGRA), Astron. Astrophys. 311 (1996) L13.
% \bibitem{ines40} F. W. Stecker, C. Done, M. H. Salamon and P. Sommers, Phys. Rev. Lett. 66 (1991) 2697 [Erratum-ibid. 69 (1992) 2738].
% \bibitem{AGN_neutrino1} F. Halzen y E. Zas, Astrophys. J. 488 (1997) 669.
% \bibitem{AGN_neutrino3} R. J. Protheroe, High Energy Neutrinos from Blazars, Acreation Phenomena and Related Outflows, IAU Colloq. 163, ed. D Wickramashinghe et al, 1996. 
% \bibitem{Mkr1} J. P. Rachen, Hadronic Correlated Flares from Mrk 501, gamma-2000 conference, Heidelberg, astroph/0010289.
% \bibitem{Mkr2} G. Ghisellini, Spectra and power of blazar jets, conference X-ray Astronomy 2000, Palermo, Italia, Sept. 2000, astroph/0012125.
% \bibitem{Desc_GRB} R.W. Klebesadel, I.B. Strong, R.A. Olson, Astrophys. J. 182 (1973) L85.
% \bibitem{Observatorio_GRB1} W.S. Paciesas et al., Astrophys. J. Suppl. 122 (1999) 465.
% \bibitem{Observatorio_GRB2} F. Frontera, in: Proceedings of GRBs in the Afterglow Era 2002, ASPConference Series, vol. 312, 2004, p. 3; M. De Pasquale et al, preprint astro-ph/0507708.
% \bibitem{Actual_GRB1} R. Vanderspek, HETE collaboration, AAS 37 (2005) 1410.
% \bibitem{Actual_GRB2} C. Winkler, New Astron. Rev. 50 (2006) 530.
% \bibitem{Actual_GRB3} N. Gehrels et al., ApJ 611 (2004) 1005.
% \bibitem{Actual_GRB4} M. Tavani et al., in: Proceedings of SPIE, Orlando, Florida, 24–31 May, vols. 6265–6267, 2006.
% \bibitem{AltasEnergias_GRB1} M.M. Gonzales, B.L. Dingus, Y. Kaneko, R.D. Preece, C.D. Dermer, M.S. Briggs, Nature 424 (2003) 749.
% \bibitem{AltasEnergias_GRB2} K. Hurley et al., Nature 372 (1994) 652.
% \bibitem{AltasEnergias_GRB3} R.W. Atkins, Astrophys. J. 583 (2003) 824.
% \bibitem{Teo_GRB1} Z.G. Dai, T. Lu, Astrophys. J. 580 (2002) 1013.
% \bibitem{Teo_GRB2} S. Razzaque, P. Meszaros, B. Zhang, Astrophys. J. 613 (2004) 1072.
% \bibitem{Teo_GRB3} B. Zhang, P. Meszaros, Astrophys. J. 559 (2001) 110.
% \bibitem{Teo_GRB4} M. Böttcher, C.D. Dermer, Astrophys. J. 499 (1998) L131.
% \bibitem{Wax_Bah} E. Waxman y J. Bahcall, Phys. Rev. D 59 (1999) 023002-1
% \bibitem{Sin_Consenso} J. P. Rachen, R. J. Protheroe y K. Mannheim, The relation of extra-galactic cosmic rays and neutrino fluxes: the logic of the upper bound debate, 19 Texas Symposium, Paris, December, astroph/9908031.
% \bibitem{MPR} K. Mannheim, R. J. Protheroe y J. P. Rachen, Phys. Rev. D 63 (2000)
% \bibitem{hipotesis_WB} E. Waxman y J. Bahcall, Phys. Rev. D 64 (2001) 
% \bibitem{Neutrinos_Atmosfericos} P. Lipari, Astropart. Phys. 1 (1993) 195.
% \bibitem{Exotica1} P. Bhattacharjee et al, Phys Rev. Lett. 69 (1992) 567.
% \bibitem{Exotica2}G. Sigl et al, Phys. Lett. B392 (1997) 129
% \bibitem{Exotica3} U. Wichoski et al, High Energy Neutrinos, Photons and Cosmic Rays from Non-Scaling Cosmic strings, hepph/9805419.
% \bibitem{ines42} K. S. Hirata et al. [Kamiokande-II Collaboration], Phys. Rev. D 44 (1991) 2241 [Erratum-ibid. D 45 (1992) 2170].
% \bibitem{ines43} K. Hirata et al. [KAMIOKANDE-II Collaboration], Phys. Rev. Lett. 58 (1987) 1490.


% \bibitem{ines47} J. D. Zornoza [IceCube Collaboration], Nucl. Phys. Proc. Suppl. 165 (2007) 196.
% \bibitem{ines50} R. Coniglione [NEMO Collaboration], Nucl. Phys. Proc. Suppl. 168 (2007) 271.
% \bibitem{ines4} R. Abbasi et al. [HiRes Collaboration], arXiv:astro-ph/0703099.
% \bibitem{ines23} J. Matthews, Astropart. Phys. 22 (2005) 387.
% \bibitem{ines24} T. K. Gaisser,Cosmic Rays and Particles Physics.
% \bibitem{ines25} M. V. S. Rao and B. V. Sreekantan, Extensive Air Showers.
% \bibitem{ines26} K. Greisen, Prog. in Cosmic Rays Phys. ed. J. G. Wilson, Vol. III, p.1, North Holland Publ. Co. (1656).
% \bibitem{ines27} V. Avati et al., Eur. Phys. J. C 34 (2004) S255.
% \bibitem{ines28} G. Anelli et al. [TOTEM Collaboration], arXiv:hep-ex/0602025.
% \bibitem{ines29} T. K. Gaisser and A. M. Hillas, Proc. 15th I.C.R.C., Plovdiv, Bulgaria, 8 (1977) 353.
% \bibitem{ines30} L. Cazón, Ph. Thesis Modeling the muón time distribution in extensive air showers, 2004.
% \bibitem{ines32} M. Ave, R. A. Vazquez, E. Zas, J. A. Hinton and A. A. Watson, Astropart. Phys. 14 (2000) 109.
% \bibitem{Aires} S. Sciutto, {\sc Aires} a system for air shower simulation, version 2.8.4a, 2006. URL: <http://www.fisica.unlp.edu.ar/auger/aires/>.
% \bibitem{ines33} P. Billoir and P. Sommers, Comptes Rendus Physique 5 (2004) 495.
% \bibitem{ines34} L. Cazón, R. A. Vazquez and E. Zas, Astropart. Phys. 23 (2005) 393.
% \bibitem{ines35} R. Mirzoian et al., Nucl. Instrum. Meth. A 351 (1994) 513.
% \bibitem{ines36} T. C. Weekes, AIP Conf. Proc. 558 (2001) 15.
% \bibitem{ines37} K. Green, J. L. Rosner, D. A. Suprun and J. F. Wilkerson, Nucl. Instrum. Meth. A 498 (2003) 256.
% \bibitem{ines38} J. L. Rosner and D. A. Suprun, AIP Conf. Proc. 579 (2001) 81.
% \bibitem{ines53} V. S. Berezinsky and A. Yu. Smirnov, Astrophys. Space Science 32 (1975) 461.
% \bibitem{Tesis_Busca} N. Busca, PhD Thesis, Diciembre 2006, University of Chicago,Department of Physics.
% \bibitem{Zas44} T. Hany D. Hooper, New J. Phys 6 (2004) 150 (arXiv:hep-ph/0408348).
% \bibitem{ines51} R. Gandhi, C. Quigg, M. H. Reno and I. Sarcevic, Phys. Rev. D 58 (1998) 093009.
% \bibitem{ines52} E. Zas, New J. Phys. 7 (2005) 130.
% \bibitem{zas_nu_detection} E. Zas, ``Neutrino detection with inclined air showers'', New Journal of Physics {\bf 7} 130 (2005).
% \bibitem{ines54} S. Yoshida et al., ``A Search For Horizontal Air Showers Induced By Extremely High Energy Cosmic Neutrinos Observed By Akeno Giant Air Shower Array'' Prepared for 27th International Cosmic Ray Conference (ICRC 2001), Hamburg, Germany (2001).
% \bibitem{ines55} M. Ave, R. A. Vazquez, E. Zas, J. A. Hinton and A. A. Watson, ``Near horizontal showers detected with the Haverah Park Array'', Prepared for 26th International Cosmic Ray Conference (ICRC 99), Salt Lake City, Utah (1999).    
% \bibitem{Zas16} E. Zas, F. Halzen y R.A. Vazquez, Astropart. Phys 1(1993) 297.
% \bibitem{Zas60} G. Parente y E.Zas, Proc. XXV Int. Cosmic Ray Conf. on High Energy Physics (HEP 95), Bruselas; Proc. XXV Int. Cosmic Ray Conf. (Durban 1997), vol 7, pp. 109-112.
% 
% \bibitem{CTEQ6} See for instance ``The Coordinated Theoretical-Experimental Project on QCD (CTEQ)'' web page at http://www.phys.psu.edu/$\sim$cteq/ and references therein. 
% \bibitem{Oscilaciones_Cosmologicas} H. Athar et al., Phys.Rev. D 62 103007 (2000).
% H. Athar et al., hep-ph/005104v2
% O. Yasuda, hep-ph/005135v1
% \bibitem{Primeros_Nuetrinos_Solares} D. Davis, Jr., D. S. Harmer y K.C. Hoffman, Phys. Rev. Lett. 20, 1205 (1968).
% \bibitem{Oscilaciones_DATAGROUP} C. Amsler et al. (Particle Data Group), Physics Letters B667, 1 (2008)
% \bibitem{Xavier_tesis} X. Bertau, PhD Thesis, November 2001, Université de Paris - Denis Diderot, UFR de Physique.
% \bibitem{ines57} M. M. Guzzo and C. A. Moura, arXiv:hep-ph/0312119.
% \bibitem{Zas34} F. Halzen and D. Saltzberg, Phys. Rev. Lett. 81 (1998) 4305.
% \bibitem{ines59} X. Bertou, P. Billoir, O. Deligny, C. Lachaud and A. Letessier-Selvon, Astropart. Phys. 17 (2002) 183.
% \bibitem{ines61} J. Abraham et al. [Pierre Auger Collaboration], Nucl. Instrum. Meth. A 523 (2004) 50.
% \bibitem{auger_prop04} Auger Collaboration, Nuclear Instruments \& Methods in Physics Research, A523, 50-95, (2004)
% \bibitem{icrctri} D. Allard y otros, ICRC, (2005)
% \bibitem{gap-05-05} P. L. Ghia, GAP Note 5, (2005)
% \bibitem{FDtrigger} G. Espenlaub, GAP Note 43, (1996)
% \bibitem{ROOT} http://root.cern.ch
% \bibitem{Billoir_resampling} P. Billoir, ``Reconstruction of Showers with the Ground Array: Status of the Prototype Program'' GAP-2000-025.
% \bibitem{HAS_spectrum_USC} P. Facal San Luis, V.M. Olmos-Gilbaja, G. Parente, G. Rodr\'\i guez-Fern\'andez, I. Vali\~no, R.A. V\'azquez and E. Zas, ``Cosmic ray spectrum with inclineds showers'', GAP-2007-027.   
% \bibitem{tesis_Ines} I. Vali\~no, PhD Thesis, December 2007, Univ. Santiago de Compostela, Spain.
% \bibitem{EMhalo} X. Bertou, GAP-2002-019; S. Coutu, X. Bertou, P. Billoir, GAP-1999-030
% \bibitem{Jaime_ICRC07} J. Alvarez-Mu\~niz for the Pierre Auger Collaboration, Proceedings of the XXX International Cosmic Ray Conference ICRC 2007, M\'erida, M\'exico. arXiv:0706.4322
% \bibitem{Facal_ICRC07} P. Facal for the Pierre Auger Collaboration, Proceedings of the XXX International Cosmic Ray Conference ICRC 2007, M\'erida, M\'exico. arXiv:0706.4322
% \bibitem{Parizot_GAP} I. Lhenry, E. Parizot, D. Allard, P. Ghia, J. Aublin ``The Lateral Trigger Probability function (LTP) for UHE showers detected by the PAO surface detector'' GAP-2003-088.
% \bibitem{Piera} C. Bonifazi and P. L. Ghia, ``Selection of data periods and calculation of the SD geometrical acceptance'', GAP-2006-101
% \bibitem{hires} K. Martens [HiRes Collaboration], arXiv: 0707.4417
% \bibitem{rice} I. Kravchenko {\it et al.} [RICE Collaboration] Phys. Rev. D {\bf 73} 82002 (2006). 
% \bibitem{ANITA} S.W. Barwick {\it et al.} [ANITA Collaboration] Phys. Rev. Lett. {\bf 96} 171101 (2006).
% \bibitem{auger_up} J. Abraham et al. [Pierre Auger Collaboration] ``Upper limit on the diffuse flux of UHE tau neutrinos from the Pierre Auger Observatory''.Accepted in Phys. Rev. Lett. (2008).
% \bibitem{Auger_North} http://www.auger.org/admin/augernorth/AN-Design-Report-r28.pdf
    
%Citas sacadas de ines

%\bibitem{ines18} E. Fermi, Phys. Rev. 75 (1949) 1169.
%\bibitem{ines19} R. D. Blandford and J. P. Ostriker, Astrophys. J. 221 (1978) L29.
%\bibitem{ines20} A. R. Bell, Mon. Not. Roy. Astron. Soc. 182 (1978) 147.
%\bibitem{ines21} Y. A. Gallant and A. Achterberg, arXiv:astro-ph/9812316.




%Citas sacadas de Javier

%\bibitem{coc22} G. Cocconi, Nuovo Cimento, 3, 1422
%\bibitem{gre66} K. Greisen, Phys. Rev. Lett. 16, 748 (1966)
%\bibitem{zat66} G.T. Zatsepin \& V.A. Kuzmin, JETP Lett. 4 78 (1966)
%\bibitem{hei54} W. Heitler, The Quantum Theory of Radiation, third ed., Oxford University Press, London, p. 386 (Section38), (1954)

%\bibitem{icrcLDF} P. Bauleo y otros, ICRC, (2005)
%\bibitem{gap-05-59} M. Gomes Berizo y otros, GAP Note 59, (2005) 


%\bibitem{lateralMuon} P. Billoir, GAP Note 86, (2005)

%\bibitem{Cronin} J. W. Cronin, Nuclear Physics, 29B, (1992)
    
%Citas sacadas de la GAP
    

%\bibitem{Blanch_GAPS} O. Blanch and P. Billoir, GAP-2005-047, GAP-2005-066 and GAP-2006-054.
%\bibitem{Auger_photon_limit} J. Abraham {\it et al.} [Auger Collaboration] ``An upper limit to the photon fraction in cosmic rays above $10^{19}$ eV from the Pierre Auger Observatory'' Astroparticle Physics, {\bf 27}, 155-168 (2007); J. Abraham {\it et al.} [Auger Collaboration] ``Upper limit on the Cosmic-Ray Photon Flux above $10^{19}$ eV using the Surface Detector of the Pierre Auger Observatory'' Astroparticle Physics (2008) in press.
%\bibitem{Gora} D. G\'ora, M. Roth and A. Tamburro ``A MC approach to simulate up- and down-going neutrino showers including local topographic conditions'' Astroparticle Physics {\bf 26} 402 (2007).


%\bibitem{CCvsNC} %CC and NC $y$ distributions are similar
%R. Gandhi, C. Quigg, M. H. Reno, I. Sarcevic, Astroparticle Physics {\bf 5} 81 (1996); and Phys. Rev. {\bf D58} 093009 (1998).


\end{thebibliography}