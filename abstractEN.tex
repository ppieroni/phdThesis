\begin{titlepage}

$\,$\\[2.5cm]
\textbf{\Large  Search for ultra high energy neutrinos using the Surface Detector of the Pierre Auger Observatory}\\[2.5cm]

\noindent
\textbf{Abstract: } the Surface Detector of the Pierre Auger Observatory is sensitive to neutrinos of all flavours above 0.1 EeV. These
interact through charged and neutral currents in the atmosphere giving rise to extended air showers. When interacting deeply in the
atmosphere at nearly horizontal incidence, neutrinos can be distinguished from regular hadronic cosmic rays by the 
broad time structure of the signals in the water-Cherenkov detectors.
In this paper we present for the first time an analysis based on down-going neutrinos. We describe the search procedure, 
the possible sources of background, the method to compute the exposure and the associated systematic uncertainties.
No candidate neutrinos have been found in the data collected from 1 January 2004 to 31 May 2010. 
Assuming a differential flux $\Phi(E_\nu) = k\cdot E_\nu^{-2}$ in the energy range from 0.1 to 10 EeV, we place a 90\% CL 
upper limit on the single flavour neutrino flux of $k < 1.65 \times 10^{-7}~{\rm GeV~cm^{-2}~s^{-1}~sr^{-1}}$.\\[0.2cm]

\noindent
\textbf{Keywords: } astroparticles, cosmogenic neutrinos, UHE neutrinos, cosmic rays, Pierre Auger Observatory\\

\end{titlepage}