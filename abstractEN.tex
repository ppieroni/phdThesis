\begin{titlepage}

$\,$\\[2.5cm]
\textbf{\Large  Measurement of the ultra high energy cosmic neutrino flux by means of surface detectors.}\\[2.5cm]

\noindent
\textbf{Abstract: } 

This Thesis studies the measurement of ultra energetic cosmic neutrino by means of surface detectors. 
There are basically two mechanisms by which neutrinos in the EeV range may induce surface level distinguishable signals.
The first one consists in the interaction of a neutrino in the atmosphere, either through charged or neutral currents, and the subsequent production of a down-going extended air shower.
The second one is based on the interaction of an earth skimming tau neutrino in the earth crust.
This process gives rise to a tau lepton that can escape into the atmosphere and start in its decay a high angle extended air shower.
In both cases, when the shower starts near the detector, characteristics related to the presence of an electromagnetic component allow to discriminate this kind of events from the dominant hadronic cosmic ray background.

This work is divided in two parts, which address two different technologies to study atmospheric showers induced by neutrinos.
The first one deals with neutrino searches based in the interaction of shower particles with the Surface Detector of the Pierre Auger Observatory.
In particular a detailed description is presented of the identification procedure, the possible background sources, the method developed to compute the exposure and its systematic uncertanties.
Additionally, the integration of the different complementary neutrino search analyses developed in Auger is presented for the first time, which include two criteria to recognise down-going neutrinos and one for earth-skimming neutrinos, as well as the inclusion of detector ageing in the exposure calculation.
No candidate neutrinos have been found in the data collected from 1 January 2004 to 20 June 2013. 
Assuming a differential flux $\Phi(E_\nu) = k\cdot E_\nu^{-2}$ and a flavor relation of $1:1:1$, we place a 90\% CL upper limit on the single flavour neutrino flux of $k < 6.4 \times 10^{-9}~{\rm GeV~cm^{-2}~s^{-1}~sr^{-1}}$ in the energy range from ${1.0 \times 10^{17}}$ {eV} - ${2.5 \times 10^{19}}$ {eV}, the most stringent to this date.

The second part of this work studies the potential of the detection of the radio waves produced by atmospheric showers to identify those induced by neutrinos, as well as the capabilities and limitations of an antenna array to detect ultra high energy earth-skimming neutrinos.
This involved simulations of the different intervening processes, including the neutrino interaction in the Earth, the tau lepton decay in the atmosphere, the subsequent development of the shower including the production of electromagnetic radiation in radio frecuencies, and the signal production in the detector antennas.
From these simulations, reconstruction and identification algorithms were developed that allow to distinguish events initiated by neutrinos from the hadronic background.
Finally these criteria were used to compute the exposure achieved by detectors with different sizes and topologies, concluding that the technique can be a competitive option in the next generation of neutrinos detectors.\\[0.2cm]

\noindent
\textbf{Keywords: } cosmic rays, UHE neutrinos, cosmogenic neutrinos, Pierre Auger Observatory, atmospheric showers, emission and detection of radio waves\\

\end{titlepage}












