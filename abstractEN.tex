\begin{titlepage}

$\,$\\[2.5cm]
\textbf{\Large  Measurement of the ultra high energy cosmic neutrino flux by means of surface detectors.}\\[2.5cm]

\noindent
\textbf{Abstract: } 

The measurement of ultra energetic cosmic neutrino by means of surface detectors is studied in this thesis. 
There are basically two mechanisms by which neutrinos in the EeV range may induce surface level distinguishable signals.
The first one consists in the interaction of a neutrino in the atmosphere, through charged or neutral currents, and the subsequent production of a down-going extended air shower.
The second one is based in the interaction of a tau neutrino that skims the earth crust.
This process gives place to a tau lepton capable to escape to the atmosphere and start in its decay a high angle extended air shower.
In both cases, when the shower starts near the detector, there exist characteristics related to the precence of electromagnetic component in the shower, that allow to discriminate this kind of events in the dominant background due to hadronic cosmic ray showers.

In this work is divided in two parts, which address two different technologies to study atmospheric showers induced by neutrinos.
The fist part treat the neutrino searches based in the interaction between the particles of the shower and the Surface Detector of the Pierre Auger Observatory.
In particular the identification procedure, the background sources, the developed method to compute the exposure and its systematic uncertanties are described.
Additionally, by the first time the integration of the different complementary analysis developed in Auger are presented, which include two criteria to recognise down-going neutrinos and one to earth-skimming neutrino, in an only result, and the inclution of the detector ageing in the exposure calculation.
No candidate neutrinos have been found in the data collected from 1 January 2004 to 20 Jun 2013. 
Assuming a differential flux $\Phi(E_\nu) = k\cdot E_\nu^{-2}$ and a flavor relation of $1:1:1$, we place a 90\% CL upper limit on the single flavour neutrino flux of $k < 6.4 \times 10^{-9}~{\rm GeV~cm^{-2}~s^{-1}~sr^{-1}}$ in the energy range from ${1.0 \times 10^{17}}$ {eV} - ${2.5 \times 10^{19}}$ {eV}, the most stringient to this date.

In the second part of this work the potential of the detection of the radio waves produced by the atmospheric showers to identify those induced by neutrinos is studied as the capabilities and limitations of an antenna array detecting ultra high energy earth-skimming neutrinos.
To do this Monte Carlo simulation were used, including the neutrino interaction in the Earth, its decay in the atmosphere, the subsequent development of the shower, including the electromagnetic radiation in radio frecuencies, and the signal simulation in the detector antennas.
From these simulations reconstruction and identification techniqes were developed that allow to distinguish event initiated by neutrinos in the hadronic background.
Finally these criteria were used to compute the exposure achieved by detectors with different sizes and topologies, concluding that the technique can be a competitive option in the next generation of neutrinos detectors.\\[0.2cm]

\noindent
\textbf{Keywords: } cosmic rays, UHE neutrinos, cosmogenic neutrinos, Pierre Auger Observatory, atmospheric showers, emission and detection of radio waves\\

\end{titlepage}












