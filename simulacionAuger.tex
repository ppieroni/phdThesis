%\chapter{Obtención de la muestra de señal: cadena de simulaciones}
\chapter{Simulación de los eventos de neutrinos}
\label{ch:simulacionAuger}

Debido a la complejidad de la generación de la lluvia y la respuesta del detector, la \'unica manera viable de obtener una muestra de señal es mediante simulaciones.
Para ello, son necesarios una serie de programas que se encargan de procesar las distintas etapas del proceso de detecci\'on:
\begin{description}
 \item[Interacción primaria:] esta incluye la simulaci\'on de la interacción neutrino nucleón y, cuando corresponda, del decaimiento del \tauon{}. 
 Para esto se utilizan respectivamente dos programas específicos llamados \herwig{} y \tauola{}.
 \item[Desarrollo de la EAS:] una vez obtenidos los productos de la primera interacción se utiliza un programa llamado \aires{}, que simula mediante t\'ecnicas de Monte Carlo la evolución de la cascada a través de la atmósfera.
 \item[Señal en el detector:] con la lista de partículas que alcanzan al suelo es posible simular la señal que estas generarían en el detector, utilizando el software oficial de la colaboración llamado \Offline{}.
\end{description}

	\section{Interacción primaria: \herwig{} - \tauola{}}
	
	El programa \aires{} no incluye herramientas para para procesar la interacción neutrino-nucleón ni el decaimiento del \tauon{}.
	Para salvar este inconveniente fue necesario utilizar programas complementarios, que devuelven una lista de partículas y sus cuadrimomentos que pueden ser utilizadas para iniciar la EAS en \aires.
	
	\subsection{Interacción neutrino-nucleón: \herwig{}}
	
	La interacción neutrino-nucleón en la atm\'osfera fue procesada utilizando el paquete \herwig{}~\cite{cite:herwig}.
	Este generador de eventos de alta energía es capaz de realizar una simulación detallada de los procesos QCD, incluidas las lluvias partónicas generadas y su hadronización posterior.
% 	Para ello se tomaron en cuenta las interacciones via CN y la de todos los sabores via CC.

	La fracción de energía transferida por el neutrino primario a la lluvia depende en gran medida del canal de interacción débil involucrado.
	Si la lluvia es iniciada por un $\nu_e$ vía el canal de CC, el 100\% de la energía del neutrino primario es transferida.
	En contraposición, en las interacciones CN, se produce un neutrino secundario en lugar de un electrón. Este neutrino lleva, en promedio, el 80\% de la energía del primario.
	Como la probabilidad de que este neutrino secundario escape sin interactuar es muy alta, solo una fracción de la energía del primario es transferida a la cascada. 

	A las energías involucradas, las lluvias iniciadas por $\nu_e$, $\nu_{\mu}$ y $\nu_{\tau}$ vía CN son indistinguibles desde el punto de vista del jet hadrónico producido.
	Es por ello que simular uno solo de los tres sabores fue suficiente para describir los tres canales de CN.

	Las lluvias iniciadas por $\nu_{\mu}$ vía CC son muy similares a las de NC pese a que la interacción fundamental involucrada es diferente ya que, tal como se discuti\'o en la sección \ref{sc:easNu}, es muy poco probable que el muón de alta energía producido decaiga o interactúe antes de alcanzar la tierra.
	De esta manera, es indistinguible de un neutrino secundario que emerge de una interacción de CN. 
	Además, debido a que la distribución de la fracción de energía que se transmite al jet (inelasticidad) es muy similar en interacciones vía CN y CC  (ver figura \ref{fig:inelast}), el conjunto de simulaciones producido para los canales de CN puede se utilizado para describir, con excelente aproximación, el canal $\nu_{\mu}$ vía CC.
	%
	\begin{figure}[ht]
	\begin{center}
	\includegraphics[width=0.80\textwidth]{fig/simulacionAuger/Inelasticity}
	\caption{Comparación de la distribución de inelasticidad para interacciones $\nu$--nucleón de alta energía para CN y CC.}
	\label{fig:inelast}
	\end{center}
	\end{figure}
	%

	
	\subsection{Decaimiento del \tauon{}: \tauola{}}
	\label{sbsc:tauola}
	El canal $\nu_{\tau}$ vía CC es un caso que requiere un tratamiento más sofisticado tanto para eventos DG o ES.
	En ambos casos, luego de propagarse a trav\'es de la atm\'osfera, el \tauon{} decae según distintos canales, generando part\'iculas secundarias que luego dan lugar a la EAS.
	Para simular estos eventos se utiliza un paquete llamado \tauola{}.
	Este contiene las probabilidades de cada canal y mediante simulaciones Monte Carlo implementa los estados finales lept\'onicos y semilept\'onicos, entregando la topolog\'ia completa del evento, incluyendo neutrinos, resonancias intermedias y la estructura completa de spin durante el desarrollo del decaimiento.
	
	La tabla \ref{tab:tauDecay} muestra los primeros ocho canales de decaimiento del \tauon{} utilizados por \tauola{}, que representa el $\sim96\%$ de los eventos esperados.
	\begin{table}[h]
		\begin{center}
		\begin{tabular}{|c|l|l|}
		% \hline
		\hline
		\# &Canal   & Probabilidad (\%) \\
		\hline
		1&$\tau^{-}\rightarrow \pi^{-}\pi^{0}\nu_{\tau}$   & $\sim$25.5 \\
		2&$\tau^{-}\rightarrow e^{-}\bar{\nu_{e}}\nu_{\tau}$   & $\sim$17.9 \\
		3&$\tau^{-}\rightarrow \mu^{-}\bar{\nu_{\mu}}\nu_{\tau}$   & $\sim$17.4 {\bf No genera EAS}\\
		4&$\tau^{-}\rightarrow \pi^{-}\nu_{\tau}$   & $\sim$10.9 \\
		5&$\tau^{-}\rightarrow 2\pi^{-}\pi^{+}\nu_{\tau}$   & $\sim$9.3 \\
		6&$\tau^{-}\rightarrow \pi^{-}2\pi^{0}\nu_{\tau}$   & $\sim$9.3 \\
		7&$\tau^{-}\rightarrow 2\pi^{-}\pi^{+}\pi^{0}\nu_{\tau}$   & $\sim$4.6 \\
		8&$\tau^{-}\rightarrow \pi^{-}3\pi^{0}\nu_{\tau}$   & $\sim$1.0 \\
		\hline
		\end{tabular}
		\end{center}
		% }
		\caption{\label{tab:tauDecay}
		Canales mediante el \tauon{} decae el $\sim96\%$ de los casos. El canal numero 3 no se observa debido a que el $\mu$ generado en el decaimiento escapa sin generar una lluvia atmosférica.
		}
	\end{table}
	Dado que el canal número 3 no genera EAS\footnote{El muón escapa sin interactuar.} no es necesario simularlo, ya que resulta una buena aproximaci\'on considerar que estos eventos no disparan el detector.
	Por este motivo las eficiencias de trigger e identificación de eventos ES tienen un valor máximo de $\sim0.826$.
	
	\section{Desarrollo de las EAS: \aires{}}

	En la actualidad existen varios programas para simular EAS a lo largo de la atmósfera.
	Entre ellos, \aires{} y {\sc corsika} son los m\'as extendidos en la comunidad de rayos c\'osmicos.
	En este trabajo se utilizaron EAS simuladas por \aires{}~\cite{cite:aires}.

	\aires{} es un conjunto de programas y subrutinas escritas en Fortran que simulan la lluvia de partículas luego de la incidencia de una rayo cósmico primario.
	Este simulador provee un ambiente realista donde la propagación de las partículas se da teniendo en cuanta las características de la atmósfera, el campo geomagnético y la curvatura de la tierra. 
	
	Debido a que la cantidad de part\'iculas generada en una EAS se vuelve rapidamente inmanejable\footnote{Por ejemplo, una EAS cuya part\'icula primaria posee \cant{10^{17}}{eV} puede llegar a tener del orden de $10^8$ part\'iculas de energ\'ia mayor a \cant{100}{MeV}. Esto requerir\'ia alrededor de \cant{100}{GB} para almacenarlas y 10 d\'ias de tiempo de m\'aquina para simular cada una. Por lo general es necesario simular del orden de $10^3\textit{-}10^4$ EAS para hacer una comparaci\'on razonable con datos experimentales.}, es com\'un en estos programas utilizar un procedimiento estadístico de muestreo llamado {\em thinning}.
	Este consiste en propagar s\'olo un conjunto representativo de las part\'iculas de la lluvia.
	La elecci\'on de dicho set se realiza de manera de no alterar los valores medios ni las dispersiones de los observables de la lluvia.
	Una explicaci\'on del algoritmo puede encontrarse en \cite{thining}.

	En la propagaci\'on, las partículas que se tienen en cuenta son: fotones, electrones, positrones, muones, piones, kaones, mesones eta, bariones lambda, nucleones, antinucleones y núcleos de hasta Z=36.
	Los neutrinos electr\'onicos y mu\'onicos que son generados en ciertos decaimientos no son propagados.
	La part\'icula que inicia la lluvia puede ser cualquiera de las mencionadas y su energía que puede variar en el rango de \cant{10^9}{eV} a \cant{10^{21}}{eV}. 
	Tambi\'en es posible simular lluvias iniciadas por partículas primarias ``especiales'' utilizando un módulo que debe escribir cada usuario, capaz de procesar la primer interacción del primario y devolver una lista de part\'iculas tradicional que \aires{} pueda aceptar.
	De esta manera fue posible inyectar las salidas de \herwig{} y \tauola{} para iniciar la simulaci\'on de la EAS.

	Los procesos físicos m\'as importantes (desde el punto de vista probabilístico) que las lluvias de part\'iculas pueden sufrir son tenidos en cuenta en las simulaciones. 
	Estos son:

	\begin{itemize}
	\item Procesos electromagnéticos: producci\'on de pares y aniquilaciones electr\'on-positr\'on, bremsstrahlung (electrones, positrones y muones), producci\'on de pares mu\'onicos, electrones sacados de \'orbitas at\'omicas (rayos $\delta$), efectos Compton y fotoel\'ectrico, efecto Landau-Pomeranchuk-Migdal (LPM) y supresi\'on diel\'ectrica.
	\item Decaimientos de part\'iculas inestables.
	\item Procesos Hadr\'onicos: colisiones inel\'asticas hadr\'on-n\'ucleo y fot\'on-n\'ucleo, muchas veces simulados utilizando paquetes externos que implementan un modelo de interacci\'on hadr\'onico, como SIBYLL, QGSJET o QGSJET2, reacciones fotonucleares, fragmentaciones nucleares, elásticas e inelásticas.
	\item Propagaci\'on de part\'iculas cargadas: pérdidas de energ\'ia en el medio (ionizaci\'on), dispersiones múltiples de Coulomb y deflexiones geomagn\'eticas.
	\end{itemize}    

	Tambi\'en, el sistema de simulación de {\sc aires} provee una plataforma que permite hacer uso del poder de c\'alculo de las computadoras actuales:
	
	\begin{itemize}
	\item Implementa un lenguaje de comandos iniciales (IDL por {\em Input Directive Language}), que consiste en un conjunto simple de comandos que permiten un control eficiente de los par\'ametros de entrada para cada simulación. 
	\item El sistema que lleva a cabo las simulaciones es una herramienta poderosa en plataformas UNIX, ya que permite al usuario coordinar muchas simulaciones en simultáneo, controlar la evolución de un dado trabajo mientras que se está llevando a cabo, etc.
	\item El programa que administra la información de salida procesa archivos generados por el programa principal y permite obtener información relacionada con los observables físicos durante y al final de cada simulación.
	\item Finalmente, hay una librería que provee una serie de rutinas auxiliares para procesar la información generada. En particular, la información m\'as relevante es contenida en archivos comprimidos. \'Esta consiste en informaci\'on detallada de cada part\'icula que llega al piso y de la lluvia en diferentes alturas, que se registra durante la evolución de la misma.
	\end{itemize}
	
	\section{Señal en el detector: \Offline{}}
	\label{sc:offline}
	
	Una vez que la lluvia alcanza el nivel del suelo es necesario simular la respuesta del detector.
	Para ello se utiliza el programa \Offline{}, que fue desarrollado en C++, dise\~nado especialmente para cubrir los requerimentos del proyecto Auger. Por ejemplo:
	\begin{itemize}
	\item Contiene una intefaz intuitiva y f\'acil de comprender, que sigue la misma l\'ogica que el detector real.
	\item Dicha interfaz permite acceder a informaci\'on del detector u otros (archivada en diversos formatos) mediante comandos simples y estandarizados.
	\item Es posible cambiar completamente la implementaci\'on de cualquier algoritmo sin cambiar en absoluto la interfaz.
	\item Es posible simular un detector completamente heterog\'eneo en el que cada tanque o telescopio tiene identidad propia.
	\item El grado de detalle con el que se realizan las simulaciones puede variar f\'acilmente, seg\'un se necesite.
	\end{itemize}
	
	\subsection{Algoritmos dentro de \Offline{}}
	
	Si bien \Offline{} posee una gran cantidad de algoritmos que se utilizan en diferentes \'areas dentro de la colaboraci\'on, aqu\'i s\'olo se introducen unos pocos que son importantes en este an\'alisis.
	
		\subsubsection{Unthinning}
		
		Una vez que la EAS alcanza el nivel del suelo, las part\'iculas presentes se guardan por lo general en archivos binarios en diversos formatos, que dependen del simulador empleado.
		Si se ha utilizado un algoritmo de thinning, a cada part\'icula se le asigna un peso estad\'istico que hace referencia a la cantidad de part\'iculas que esta representa.
		\Offline{} no s\'olo permite leer archivos provenientes de varios simuladores (entre los que se encuentra \aires{}), sino que tambi\'en aplica un algoritmo de \textit{unthinning} que regenera las part\'iculas no simuladas para incluirlas en la se\~nal sobre el detector.
% 		Dado que el \'area instrumentada a nivel del suelo es muy chica y s\'olo unas pocas de las part\'iculas que llegan a este nivel interactúan con el detector,
% 		Por este motivo, para realizar el \textit{unthinning}, 
		Para ello, realiza lo que se denomina un \textit{m\'etodo de muestreo local}.
		Este consiste en seleccionar todas las part\'iculas dentro de ciertas zonas de muestreo, rodeando cada estaci\'on del SD e inyectar en el tanque una fracci\'on de las mismas.
		Esta fracci\'on tiene en cuenta el peso estad\'istico de cada una convolucionado con la probabilidad de ingreso en el detector.
		Luego, se las dispersa espacial y temporalmente utilizando un m\'etodo estad\'istico, que se describe en \cite{unthinning1}, distribuy\'endolas sobre las paredes del tanque.
% 		Una vez que las part\'iculas, clones de la primaria, son dispersadas se las inyecta en posiciones aleatorias de la pared del detector, teniendo en cuenta la direcci\'on de incidencia. 
		Un ejemplo de esto \'ultimo se muestra en la figura \ref{fig:unthinning_tank}.
		%
		\begin{figure}[h!]
			\begin{center}
			\includegraphics[width=0.7\textwidth]{fig/simulacionAuger/unthinning_tank}
			\caption{Distribuci\'on de part\'iculas sobre la superficie de un tanque luego de aplicar el algoritmo de unthinning.}
			\label{fig:unthinning_tank}
			\end{center}
		\end{figure}
		%
		
% 		En particular, el algoritmo de unthinning no es directamente aplicable a lluvias ES, por lo que una modificaci\'on del mismo se incluye en \Offline{} con el fin de resolver sus problemas.
		
% 		\textbf{RECTANGULOS ??}
		
		\subsubsection{Simulaci\'on del tanque}
		
		Para simular la respuesta del detector de superficie se ha incorporado en \Offline{} un m\'odulo especializado en la simulaci\'on del tanque basado en el paquete {\sc geant 4}~\cite{geant4}. 
		Este consiste en una librer\'ia de C++ que provee las herramientas necesarias para simular tanto la f\'isica como la geometr\'ia del detector.
% 		Algunas caracter\'isticas de dise\~no se exponen a continuaci\'on:
% 		\begin{itemize}
% 		\item Se procur\'o que todos los aspectos f\'isicos en esta etapa de la simulaci\'on sean lo m\'as precisos posible, mientras se trat\'o de tener buen entendimiento de su influencia en el resultado final.
% 		\item Por ser {\sc geant 4} una versi\'on m\'as sofisticada de programas anteriores, algunas de las variables del tanque fueron ajustadas de tal forma que los resultados experimentales se reproduzcan\footnote{Una de las m\'as importantes es la cantidad de fotoelectrones detectados por mu\'on vertical incidente.}.
% % 		\item La velocidad de simulaci\'on es un factor importante.
% % 		La simulaci\'on de estaciones cercanas al centro de la lluvia puede ser muy costosa debido a la gran cantidad de part\'iculas que se generan en esa regi\'on. Para evitar este inconveniente, se aplican cortes espec\'ificos en la producci\'on de fotones en el seguimiento de muones.
% 		\end{itemize}
		Algunos de los procesos f\'isicos tenidos en cuenta en la simulaci\'on del tanque son:
		%
		\begin{itemize}
		\item La simulaci\'on de la propagaci\'on de una part\'icula cargada puede incluir efectos de ionizaci\'on, producci\'on de rayos delta, scattering de Coulomb múltiple, bremsstrahlung y radiaci\'on Cherenkov.
		\item Lor fotones producidos en la radiaci\'on Cherenkov (incluyendo los producidos por los rayos delta) son sometidos a scattering de Rayleigh, absorci\'on e interacci\'on \'optica con las paredes del tanque. La polarizaci\'on del fot\'on es tenida en cuenta siempre que es relevante.
		\end{itemize}
		%
		En particular, se encuentran especialmente modeladas y optimizadas las propiedades \'opticas de Tyvek, la longitud de absorci\'on del agua y la geometr\'ia de los PMTs.
		En la figura \ref{fig:cher_tank_sim} se muestra la simulaci\'on de los fotones Cherenkov generados en el pasaje de un electr\'on de baja energ\'ia a trav\'es del tanque.
		%
		\begin{figure}[h!]
			\begin{center}
			\includegraphics[width=0.5\textwidth]{fig/simulacionAuger/cher_tank_sim}
			\caption{Simulaci\'on de fotones Cherenkov generados en el pasaje de un electron de baja energ\'ia a trav\'es del tanque.}
			\label{fig:cher_tank_sim}
			\end{center}
		\end{figure}
		%
		
		Por \'ultimo tambi\'en se simula la electr\'onica de la estaci\'on, por lo que luego de esta etapa cada tanque involucrado en un evento tiene sus trazas FADC calculadas.
		Con esta informaci\'on es posible calcular desde los niveles de disparo locales T2 hasta los CDAS T4 y T5, especificados en la secci\'on \ref{sbsc:trig_levels}.
		
	\section{Librería de eventos generada}
	\label{sc:libGen}
	
	En cada búsqueda se generó una librería de eventos que cubra todo el espacio de parámetros tal que puedan iniciarse lluvias atmosf\'ericas con posibilidad de disparar el detector.
	Sus detalles, junto con los criterios utilizados para definirlas se documentan en las siguientes secciones.
	
		\subsection{Eventos DGL}
		
		Para el canal CC se utilizaron energías fijas entre \cant{10^{17}}{eV} y \cant{10^{20.5}}{eV} con un paso logarítmico de $0.5$. 
		Dado que el canal NC sólo lleva el $20\%$ de la energía del neutrino, el rango energético fue acortado a \cant{10^{18}\ -\ 10^{20.5}}{eV}, evitando así simular eventos que no dispararían el detector.
		En canal double bang no fue tenido en cuenta, por lo que las lluvias iniciadas por \nutau{} se simularon como NC, subestimando así su eficiencia.
		En ambos canales, se simularon ángulos cenitales desde $60^\circ$ a $75^\circ$ con un paso de $3^\circ$, utilizando ángulos azimutales aleatorios entre $0^\circ$ y $360^\circ$.
		Con el fin de maximizar la eficiacia a la hora de simular se forzó a los neutrinos a interactuar en puntos de inyección de profundidad $D$, uniformemente distribuidos entre un valor máximo y un mínimo (que dependen del ángulo cenital) con un paso fijo de \cant{100}{g cm^{-2}}.
		Algunos puntos de inyección muy cercanos al suelo fueron omitidos debido a que la lluvia no llega a desarrollarse lo suficiente como para provocar un trigger.
		Para cada punto ($E_{\nu}$, $\theta$, $D$) se generaron 50 cascadas atmosféricas independientes.
		Cada una de estas se utilizó para producir cinco eventos con diferente punto de impacto en \Offline{}, dando un total de 250 eventos cuasi-independientes para cada punto de inyección.
		Todo lo anterior se resume en la tabla \ref{tab:sim_table_dgl}, mientras que la figura \ref{fig:sim_fig_dgl} muestra las profundidades simuladas como función del ángulo cenital.
		
		\begin{table}[ht]
			\begin{center}
			\includegraphics[width=0.85\textwidth]{fig/simulacionAuger/simShNavarro_1}
			\end{center}
			\caption{Resumen de las energías, ángulos cenitales y puntos de inyección para la simulación de lluvias DGL. Para cada angulo, se simuló un número determinado de profundidades de interacción entre el máximo y un mínimo especificado en la tabla, con un paso de \cant{100}{g\,cm^{-2}}. Los dos primeros bines de energía no fueron simulados para el canal NC.
			}
			\label{tab:sim_table_dgl}
		\end{table}
		%
		\begin{figure}[h!]
			\begin{center}
			\includegraphics[width=0.85\textwidth]{fig/simulacionAuger/simShNavarro_2}\\
			\caption{En linea llena se grafica la profundidad atmoférica inclinada del Observatorio Pierre Auger, que se encuentra a \cant{1400}{m} sobre el nivel del mar.
			Los circulos indican los puntos de inyección de los neutrinos (puntos de interacción) para diferentes ángulos cenitales.
			}
			\label{fig:sim_fig_dgl}
			\end{center}
		\end{figure}
		
		
		\subsection{Eventos DGH}
		
		La librería de eventos DGH fue obtenida con criterios muy similares a los utilizados en DGL, con la salvedad de que en este canal si se tuvieron en cuenta los eventos double bang.
		
		Con el fin de modelar las lluvias DB de manera fidedigna, se realizó la siguiente cadena de simulación:
	%
		\begin{enumerate}
		\item A partir de la simulación HERWIG de la primer interacción se obtuvo la energía transferida al leptón $\tau$ y las demás partículas producidas (jet hadrónico).
		\item Las partículas del jet hadrónico se inyectaron en \aires{} de manera análoga al procedimiento utilizado en la simulación de las lluvias producidas por $\nu_{e}$ vía CC o por $\nu_{X}$ vía CN. El leptón $\tau$ no se incluye en este grupo.
		\item Utilizando la energía transferida al leptón $\tau$ (dada por {\sc herwig}) se calculó la longitud de decaimiento y se generó al azar una distancia recorrida antes de decaer. La pérdida de energía en la atmósfera es despreciable, y no se tuvo en cuenta.
		\item Si el resultado del punto anterior indicaba que el decaimiento se produjo antes de que el $\tau$ alcance la tierra, se utilizó el paquete {\sc tauola}~\cite{cite:TAUOLA} para calcular las partículas producidas en el decaimiento.%\footnote{{\sc tauola} es un MC desarrollado en el ambiente de altas energías que contiene toda la información experimental sobre los canales dominantes de decaimiento del $\tau$. }.
		\item Los productos del decaimiento fueron inyectados en \aires{} a la profundidad y tiempo adecuados determinados por la distancia recorrida por el $\tau$.
		\item Se calculó la evolución conjunta de todas las partículas utilizando \aires{}.
		\end{enumerate}
		
		La tabla ~\ref{tab:sim_table_dgh} resume el conjunto de simulaciones producidas para este rango angular.
		\begin{table}[ht]
		\begin{center}
		\includegraphics[width=0.75\textwidth]{fig/simulacionAuger/sim_table.pdf}
		\end{center}
		\caption{
		Resumen, discriminado por canal de interacción, de las simulaciones de neutrinos profundos realizadas para el canal DGH.
		Para cada ángulo cenital se indica la profundidad de inyección máxima considerada (medida inclinada desde el detector) y la cantidad de puntos de inyección simulados en este rango. Las energías indicadas corresponden a la del neutrino primario.
		}
		\label{tab:sim_table_dgh}
		\end{table}
		Para cada ángulo cenital se indica la profundidad de inyección máxima considerada (medida inclinada desde el detector) y la cantidad de puntos de inyección simulados en este rango. Las energías indicadas corresponden a la del neutrino primario.
		A diferencia de DGL, para maximizar la estadística en este an\'alisis se arroj\'o cada lluvia generada tres veces sobre el detector.
		
		\subsection{Eventos ES}
		
		En el caso de los eventos ES los parámetros que identifican una lluvia atmosférica son el ángulo cenital $\theta$, el ángulo azimutal $\phi$, la energía \etau{} del \tauon{} que escapa de la Tierra, y la altura \xd{} sobre el detector a la que el \tauon{} decae.
		La energía del \tauon{} tom\'o valores de \cant{10^{16.5}}{eV} a \cant{10^{19.5}}{eV}, con paso de 0.5 logarítmico salvo, en la zona de baja energía cuyo paso fue de 0.25.
		El ángulo cenital barrió valores desde $90.111^\circ$ hasta $95.884^\circ$ con paso de $0.573^\circ$ ($\sim0.1 \,\rm rad$) y en la zona de baja energía el ángulo máximo fue en algunos casos $93.549^\circ$.
		El ángulo azimutal $\phi$ se eligió al azar entre $0^\circ$ y $360^\circ$.
		Con respecto a la altura de decaimiento \xd{} se utilizó un valor máximo que dependió de la energía del \tauon, y pasos de \cant{100}{m} o \cant{50}{m} para la zona de baja energía.
		Al igual que para los eventos DGH, para generar la libreria de eventos ES se simularon 50 eventos para cada punto ($\theta$,\etau{},\xd{}) y se utilizó cada lluvia 3 veces sobre el detector, logrando 150 eventos cuasi-independientes en cada caso. 
		La tabla \ref{tab:sim_table_es} condensa toda la información sobre las simulaciones ES, mientras que la figura \ref{fig:sim_fig_es} muestra los conjuntos de ($\theta$,\etau{},\xd{}) simulados.
		%
		\begin{table}
			\begin{center}
			\scriptsize
				\begin{tabular}{|c|c|cc|cc|}
				\hline
				\makebox[0.1\textwidth][c]{Grupo} & \makebox[0.13\textwidth][c]{\etau{} [eV]}&
				\makebox[0.15\textwidth][c]{\tita{} Min - Max} & \makebox[0.07\textwidth][c]{paso [º]} & \makebox[0.13\textwidth][c]{\xd{} Min - Max} & \makebox[0.07\textwidth][c]{paso [m]}\\
				\hline
				\hline
				\multirow{2}{*}{E003EeV} & \multirow{2}{*}{$3.16\times10^{16}$} & $95.884\text{ - }90.111$ & $0.573$ & $0\text{ - }600$ & $100$\\
				& & $93.549\text{ - }90.111$ & $0.573$ & $0\text{ - }300$ & $50$\\
				
				\multirow{2}{*}{E006EeV} & \multirow{2}{*}{$5.62\times10^{16}$} & \multirow{2}{*}{$93.549\text{ - }90.111$}  & \multirow{2}{*}{$0.573$} & \multirow{2}{*}{$0\text{ - }300$} & \multirow{2}{*}{$50$}\\
				& & & & &\\
				
				\multirow{2}{*}{E01EeV} & \multirow{2}{*}{$10^{17}$} & $95.884\text{ - }90.111$ & $0.573$ & $0\text{ - }700$ & $100$\\
				& & $93.549\text{ - }90.111$ & $0.573$ & $0\text{ - }300$ & $50$\\
				
				\multirow{2}{*}{E02EeV} & \multirow{2}{*}{$1.78\times10^{17}$} & \multirow{2}{*}{$93.549\text{ - }90.111$}  & \multirow{2}{*}{$0.573$} & \multirow{2}{*}{$0\text{ - }300$} & \multirow{2}{*}{$50$} \\
				& & & & &\\
				
				\multirow{2}{*}{E03EeV} & \multirow{2}{*}{$3.16\times10^{17}$} & $95.884\text{ - }90.111$ & $0.573$ & $0\text{ - }1000$ & $100$\\
				& & $93.549\text{ - }90.111$ & $0.573$ & $0\text{ - }300$ & $50$\\
				
				\multirow{2}{*}{E06EeV} & \multirow{2}{*}{$5.62\times10^{17}$} & \multirow{2}{*}{$93.549\text{ - }90.111$}  & \multirow{2}{*}{$0.573$} & \multirow{2}{*}{$0\text{ - }300$} & \multirow{2}{*}{$50$}\\
				& & & & & \\
				
				\multirow{2}{*}{E1EeV} & \multirow{2}{*}{$10^{18}$} & $95.884\text{ - }90.111$ & $0.573$ & $0\text{ - }1400$ & $100$ \\
				& & $93.549\text{ - }90.111$ & $0.573$ & $0\text{ - }300$ & $50$ \\
				
				\multirow{2}{*}{E3EeV} & \multirow{2}{*}{$3.16\times10^{18}$} & \multirow{2}{*}{$95.884\text{ - }90.111$}  & \multirow{2}{*}{$0.573$} & \multirow{2}{*}{$0\text{ - }1300$} & \multirow{2}{*}{$100$}\\
				& & & & & \\
				
				\multirow{2}{*}{E10EeV} & \multirow{2}{*}{$10^{19}$} & \multirow{2}{*}{$95.884\text{ - }90.111$}  & \multirow{2}{*}{$0.573$} & \multirow{2}{*}{$0\text{ - }1300$} & \multirow{2}{*}{$100$} \\
				& & & & & \\
				
				\multirow{2}{*}{E30EeV} & \multirow{2}{*}{$10^{19.5}$} & \multirow{2}{*}{$95.884\text{ - }90.111$}  & \multirow{2}{*}{$0.573$} & \multirow{2}{*}{$0\text{ - }2500$} & \multirow{2}{*}{$100$} \\
				& & & & &\\
				\hline
				\end{tabular}
			\end{center}
			\caption{\label{tab:sim_table_es}
			Detalle de los parámetros utilizados para generar la librería de eventos ES.
			}
		\end{table}
		%
		\begin{figure}[h!]
			\begin{center}
			\includegraphics[width=0.65\textwidth]{fig/simulacionAuger/3D_completeParameterSpace5}
			\caption{
			Conjunto de ($\theta$,\etau{},\xd{}) utilizados para generar la librería de eventos ES.
			}
			\label{fig:sim_fig_es}
			\end{center}
		\end{figure}
		
	\section{Pesos de los eventos simulados}
	\label{sc:pesos}
	
	Si bien hasta el momento cada punto del espacio de parámetros, $(E_\nu,\theta,\phi,D)$ en DG y $(E_\tau,\theta,\phi,x_d)$ en ES, se encuentra bien representado, la frecuencia relativa de ocurrencia de eventos con tales par\'ametros no es la elegida a la hora de simular.
	Por este motivo, antes de entrenar el método de selección será necesario corregir el peso estad\'istico de la muestra.
		
		\subsection{Correción en DG}
		
		Para aplicar esta correcci\'on es necesario identificar todos los factores que modifican la cantidad de eventos en funci\'on de $E_\nu$, $\theta$, $\phi$ y $D$.
		En primer lugar, tal como se discuti\'o en el cap\'itulo \ref{ch:intro}, se espera que el flujo difuso de neutrinos tenga la forma $\frac{dN}{dE_\nu}=k\cdot E_\nu^{-2}$. 
		Asimismo, la sección eficaz $\nu$-nucleón crece como $E_\nu^{1/3}$~\cite{cite:cooper_sarkar}, lo que hasta el momento da una dependencia de $E_\nu^{-5/3}$.
		Sin embargo, dado que la energ\'ia se dividi\'o en bines logar\'itmicos, el volumen del espacio de par\'ametros representado por cada set de lluvias no es constante.
		Esto se corrige agregando el jacobiano $E_\nu/\log{10}$ a la ecuaci\'on, lo que deja una dependencia de $\propto E_\nu^{-2/3}$.
		Por otro lado, el peso var\'ia con el ángulo cenital seg\'un la proyección del área del detector en la dirección del flujo incidente (ver sección \ref{sc:expoNu} más adelante para una explicación detallada), de la fracci\'on de \'angulo s\'olido representada y de que el paso en los ángulos simulados no es constante, agregando un factor $\cos\theta \sin\theta\Delta\theta$ a la ecuaci\'on.
		Despreciando la influencia del campo geomagn\'etico, los pesos no var\'ian con el \'angulo azimutal $\phi$.
		Por último, dado que la sección eficaz de los neutrinos es pequeña, su rango en la atmósfera es varios \'ordenes de magnitud m\'as grande que su espesor.
		Por este motivo resulta una buena aproximación considerar que todos las profundidades $D$ son igual de probables.
		Uniendo los ingredientes anteriores se obtiene:
		%
		\begin{equation}
		w(E_\nu,\theta) \propto \cos\theta \sin\theta \, E_\nu^{-2/3} \Delta\theta
		\end{equation}
		%
		
		\subsection{Corrección en ES}
		\label{sbsc:corrES}
		En este caso los parámetros que nos interesa corregir son $(E_\tau,\theta,\phi,x_d)$.
		Si bien la dependencia con el ángulo sólido se calcula de la misma manera que en eventos DG, la inclusi\'on de la probabilidad de que el \tauon{} escape de la tierra con cierta energía y decaiga a cierta altura complica significativamente el c\'alculo.
		Esto puede verse en la ecuaci\'on \ref{eq:wi}, en la que se escribe la integral que hay que resolver para calcular los pesos $w(E_\tau,\theta,x_d)$.
		%
		\begin{equation}
		\begin{aligned}
		 w(E_\tau,\theta,x_d) \propto  
		 \left[
		 \int_{E_\tau}^\infty \frac{dN}{dE_\tau}(E_\nu) f(E_\tau|E_\nu,\theta) dE_\nu
		 \right]
		 h(x_d,(E_\tau,\theta)) E_\tau 
		 \cos\theta \sin\theta&\\
		 \Delta\log E_\tau \Delta x_d \Delta\theta&
		 \end{aligned}
		 \label{eq:wi}
		\end{equation}
		%
		La aparición de los términos $E_\tau \Delta\log E_\tau \Delta x_d$ se debe nuevamente a que los bines simulados no se encuentran uniformemente distribuidos en las variables $(E_\tau,x_d)$.
		Al igual que para las lluvias DG, el flujo difuso de neutrinos se supone $\frac{dN}{dE_\tau}(E_\nu)=k\cdot E_\nu^{-2}$ y la contribuci\'on del \'angulo s\'olido es $\cos\theta \sin\theta$.
		La función $h({\rm x_d},(E_\tau,\theta))$ corresponde a la probabilidad de que un \tauon{} de parámetros $(E_\tau,\theta)$ decaiga a una altura \xd{} del detector, y tiene la forma (distribuci\'on exponencial):
		%
		\begin{equation}
		 h({\rm x_d},(E_\tau,\theta))=
		 \exp{\left(
		 -\frac{l({\rm x_d},\theta)}{\lambda(E_\tau)}
		 \right)}
		 \frac{dl}{dx_d}\frac{1}{\lambda(E_\tau)}
		\end{equation}
		%
		donde $\lambda(E_\tau)=c\uptau\frac{E_\tau}{m_\tau}\sim49\left(\frac{E_\tau}{\rm EeV}\right){\rm km}$ y $l({\rm x_d},\theta)=\frac{{\rm x_d}}{|\cos\theta|}$, quedando:
		\begin{equation}
		 h({\rm x_d},(E_\tau,\theta))=
		 \exp{\left(
		 -\frac{{\rm x_d}}{|\cos\theta|\lambda(E_\tau)}
		 \right)}
		 \frac{1}{|\cos\theta|\lambda(E_\tau)}
		 \label{eq:ptaudecay}
		\end{equation}
		%
		Por \'ultimo, en la ecuación \ref{eq:wi} la parte complicada reside en $f(E_\tau|E_\nu,\theta)$, que corresponde a la probabilidad de que dado un neutrino de energía $E_\nu$ y ángulo $\theta$ emerja de la tierra un \tauon{} de energía \etau{}.
		Para obtener estas probabilidades es necesario simular todo el proceso, lo que se desarrolla en la siguiente secci\'on.
		
		\subsubsection{\label{sbsbsc:sim_prop_tierra}Propagaci\'on dentro de la tierra}
	
		Para simular la interacción del \nutau{} y propagaci\'on del \tauon{} en la tierra se utiliz\'o un programa que realiza el c\'alculo mediante m\'etodos de Monte Carlo. Este fue desarrollado por J. A. Mu\~niz y E. Zas, del grupo de altas energ\'ias de la Universidad de Santiago de Compostela.
		Una explicaci\'on detallada del programa puede encontrarse en \cite{gap_tau_tierra}.
		
		Este c\'odigo permite propagar los \nutau{}'s y \tauon{}'s en la tierra y, dado el \'angulo cenital \tita{} y la energ\'ia del neutrino incidente \enu{}, calcular la probabilidad de que un \tauon{} alcance la superficie de la tierra con energ\'ia \etau{}.
		Estos c\'alculos se realizan teniendo en cuenta la interacci\'on del \nutau{} via CC y CN, la p\'erdida de energ\'ia del \tauon{} (con $\beta$
% 		\footnote{ver ecuaci\'on \ref{eq:tau_losses}}
		dependiente o independiente de la energ\'ia) y la regeneraci\'on del flujo de \nutau{} debido a interacciones via CN y a decaimientos de leptones \tauon{}.
		
		La simulaci\'on consta de dos etapas principales que se alternan hasta que se alcanza la superficie:
		%
		\begin{description}
		\item[\textbf{Interacci\'on del \nutau{}:}] Se hace interactuar al neutrino siguiendo una distribuci\'on exponencial en $x$, la variable que mide la cantidad de materia atravesada por el \nutau{}. Para ello, se tienen en cuenta las secciones eficaces CC y CN. Si se diera una interacci\'on via CC, se produce un \tauon{} cuya energ\'ia proviene de muestrear la sección efic\'az diferencial de CC, $\frac{d\sigma^{cc}}{dy}$. En cambio, si la interacci\'on se produce via CN, es generado un nuevo \nutau{} de energ\'ia menor que se hace interactuar nuevamente.
		\item[\textbf{Propagaci\'on del \tauon{}:}] En caso de que un \tauon{} sea producido, se lo propaga de a peque\~nos pasos, calculando en cada uno de ellos la energ\'ia perdida debido a la interacci\'on con el medio y la probabilidad de decaimiento (como funci\'on de su energ\'ia). Si decayese produciendo un nuevo \nutau{}, se comienza el procedimiento nuevamente desde el punto de decaimiento.
		\end{description}
		%
		Este proceso se esquematiza en la figura \ref{fig:tau_sim_sch}.
		\begin{figure}[ht!]
			\begin{center}
			\includegraphics[width=0.75\textwidth]{fig/simulacionAuger/schemeEarthProp}
			\caption{\label{fig:tau_sim_sch} Esquema del algoritmo de simulación de las interacción en la tierra. El \nutau{} (línea sólida) y el \tauon{} (línea punteada) son simulados hasta que abandonan la tierra. La salida del algoritmo es un histograma de energía de los \tauon{} salientes.}
			\end{center}
		\end{figure}
		
		El programa fue validado utilizando otros m\'as complejos, como ANIS \cite{anis}.
		Tambi\'en se realiz\'o una comparaci\'on del flujo de \tauon{} saliente obtenido mediante el calculo te\'orico que puede encontrarse en \cite{prop_tau} y el obtenido por este programa, para diferentes \'angulos cenitales.
		Esta \'ultima comparaci\'on se muestra en la figura \ref{fig:comp_tau_mc_teo}, en la que se observa un muy buen acuerdo entre ambos resultados.
		%
		\begin{figure}[ht!]
			\begin{center}
			\includegraphics[width=0.55\textwidth]{fig/simulacionAuger/comp_tau_mc_teo}
			\caption{\label{fig:comp_tau_mc_teo} Comparaci\'on de los flujos salientes obtenido anal\'iticamente y mediante simulaci\'on. Los histogramas corresponden a los resultados del Monte Carlo, mientras que la l\'inea llena al c\'alculo te\'orico.}
			\end{center}
		\end{figure}
		
		
		Una vez obtenidos todos los términos de la ecuación \ref{eq:wi}, la integral se calcula numericamente.
