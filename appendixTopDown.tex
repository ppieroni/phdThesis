\chapter{Reconstrucción de un frente plano}
\label{ap:topDown}

En este apéndice se repasa la reconstrucción de los parámetros geométricos de la lluvia a partir de ajustar los tiempos de trigger con un frente plano.

\begin{enumerate}
\item Se calcula el baricentro:Calculate the barycenter, according to:
\begin{equation}
\boldsymbol{r_{b}} =\frac{\sum \boldsymbol{r_{i}}S_{i}^{\frac{1}{3}}}{\sum S_{i}^{\frac{1}{3}}}
\end{equation}
donde $\boldsymbol{r_{i}}$ y $S_{i}$ son la ubicacion sobre el suelo y la señal de la estación $i$.
De aquí en más , $\boldsymbol{r_{i}}$ son las coordenadas relativas a este baricentro.
\item Se ajusta un frente plano que se desplaza a la velocidad de la luz a los tiempos de trigger $t_i$, despreciando las variaciones de altura $z_{i}$ que pudiese haber respecto del suelo:
\begin{equation}
c(t_{i} -t_{0} )=-\boldsymbol{a}\boldsymbol{r_{i}} \Longleftrightarrow \langle t_{i}\rangle = t_{0} - \frac{1}{c}(ux_{i} + vy_{i} )
\end{equation}
donde $\boldsymbol{a} = (u, v, w)^{T}$ es la dirección de arrivo de la lluvia.
los parámetros libres del modelo son $u$, $v$, y $t_{0}$.
La solución analítica para obtener los parámetros del modelo se consigue mediante cuadrados mínimos.

\item Los parámetros se mejoran aplicando el mismo procedimiento, pero incluyendo la altitud de las estaciones. 
\begin{equation}
\langle t_{i}\rangle = t_{0} - \frac{1}{c}(ux_{i} + vy_{i} +w_{0}z_{i})
\end{equation}
donde $w_{0} = \max{(0, 1 - u_{0}^2 - v_{0}^{2} )}$  es la nueva componente de la dirección de la lluvia.
Dado que $w_{0}$ no es un parámetro libre, nuevamente se puede utilizar la solución de cuadrados mínimos.
Así los parámetros de la lluvia se obtienen como:
\begin{equation}
\theta = \arccos w_{0}
\end{equation}
y
\begin{equation}
\Phi = \arctan \frac{v}{u}
\end{equation}



% \item If the number of stations is 3, accept the event and refine the fit further by adding an approximate term for the spherical shape of the
% shower front:
% \begin{equation}
% <t_{i}> = t0 − \frac{1}{c}(ux_{i} + vy_{i} +w_{1}z_{i} + w_{1}\frac{r_{i}^{2}}{2d_{max}})
% \end{equation}
% where $r_{i} = x_{i} + y_{i} − (ux_{i} + vy_{i})^{2}$ is the radial distance to the preliminary shower axis
% going through the barycenter and $w_{1}$ is the fixed value from the second fit. The radius of the
% shower front sphere is roughly approximated here by $d_{max} ≈ \frac{7.1 km}{\cos{\theta}}$, where $\cos{\theta}=1 -u^{2}-v^{2}$.
% \end{enumerate}
% \item If the number of stations higher than 3, accept the current configuration under the following conditions:
% \begin{itemize}
% (a) Physical values for u and v: u2 + v 2 < 1.
% (b) Small residuals ∆ti = |ti − ti |:
% max(∆ti ) < (n − 2) min(w, 0.2) × 250 ns
%% be too large:
% √
% max(ri ) < n − 2 × 1.3 km.
% Again, the restriction is relaxed for larger showers.
% (d) Configuration may not be aligned. The principal axes of the configuration are calculated in the
% ground plane. The length of both axes has to be larger than 0.3 km. 2
% i (∆ti )
% n−3
% < (n − 2) min(w, 0.2) × 200 ns
% The factor (n − 2) relaxes the restriction in case of large events, taking into account that the
% shower front model is only approximate. The factor min(w, 0.2) tightens the restrictions as
% the shower inclination grows, because the natural variance of the arrival time is smaller in very
% inclined showers as discussed in Chapter 3.
% (c) Compact spatial configuration. The radial distance ri to the preliminary shower axis may not

\end{enumerate}