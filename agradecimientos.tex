\chapter*{Agradecimientos}

\begin{flushright}
\emph{``Cada hombre es lo que hace con lo que hicieron de \'el.''}\\
Jean-Paul Sartre\hspace*{2mm}
\end{flushright}
\vspace{1cm}
Sin dudas nunca podr\'ia haber pasado por esta etapa de mi vida sin el apoyo, la ayuda, la comprensi\'on y la orientaci\'on de muchas personas, a las que les tengo el m\'as profundo agradecimiento. Les pido disculpas por adelantado a los que pueda estar olvidando.

Fernanda, mi compa\~nera de vida, no me alcanzan las palabras ni los actos para agradecerte lo que hiciste a trav\'es de estos a\~nos. Tu apoyo incondicional y tu paciencia hicieron que esto sea posible. Por fin llegamos~\smiley{}.

Mis directores, Ricardo Piegaia y Jaime \'Alvarez-Mu\~niz, mi gratitud hacia ustedes es tambi\'en infinita, los considero dos grandes ciernt\'ificos y personas.
Durante estos a\~nos me transmitieron ense\~nanzas fundamentales para mi formaci\'on, como la importancia que tiene entender los problemas de manera intuitiva y el trabajo duro. 
% transmitieron aptitudes fundamentales para mi formaci\'on 
Agradezco haber tenido la posibilidad de impregnarme con su conocimiento y personalidad.

Tambi\'en tuve el agrado de tener como compa\~neros a dos personas que aprecio much\'isimo, Javier Tiffenberg y Yann Guardincerri. Con ellos pude compartir no s\'olo el d\'ia a d\'ia sino adem\'as viajes y experiencias que voy a recordar de por vida.
Javi, en el poco tiempo que trabajamos juntos me transmitiste formas muy particulares de ver las cosas, gracias por eso (y por ense\~narme C++!).
Yann, fue genial haber hecho equipo con vos. Me ense\~naste la importancia que tiene ser sistem\'atico en el trabajo.

Otra persona que me acompa\~n\'o espor\'adicamente pero con un impacto inmenso en este doctorado es Matias Tueros.
Probablemente no recuerde que all\'a por el 2013, en un subte de Rio, plant\'o la semilla que dio lugar a la segunda parte de esta Tesis. Alg\'un d\'ia volveremos a comer Galeto.

Mi paso por la colaboraci\'on Pierre Auger tambi\'en fue un proceso muy enriquecedor. Conoc\'i una gr\'an cantidad de personas muy capaces, que influyeron en mi manera de pensar y en mi forma de trabajar.
En particular compart\'i buenos momentos y discusiones con Hernan Whalberg, Mathieu Tartare, Enrique Zas, Jose Luis Navarro, Pierre Billoir, Piera Ghia y Esteban Roulet.

Otro pilar fundamental en esta etapa fue mi familia, que me brind\'o todo el apoyo necesario para seguir adelante.
Homero, con tus genialidades me asombr\'as constantemente y llen\'as de alegr\'ia mis d\'ias. A vos y a mam\'a les digo gracias por ser parte de mi vida.
A mis pap\'as, Hector y Leticia, que me guiaron hasta ser quien soy de la mejor manera posible, con amor, apoyo y respeto; a mis hermanos Diana y Dar\'io, y a mis t\'ios Ra\'ul y Eremita, a todos les digo simplemente gracias por existir.

Tambi\'en tuve la suerte de hacer este recorrido con un gr\'an grupo de amigos...
Agustin, que espera que termine desde que empec\'e la licenciatura.
Andrea, mi amiga de la vida.
Fernando, que me recuerda siempre que las cosas se pueden vivir de otra manera, y que en Estructura 3 no quise hacer un TP con \'el.
Julian y Nahuel, a quienes tuve la suerte de encontrar cuando m\'as necesitaba.
Andr\'es, con el que compart\'i muchos aspectos de esta magia que es la vida.
Sabrina, con la que me hubiese gustado compartir m\'as aventuras.
Mechi, que permiti\'o que pruebe el caf\'e m\'as rico que tom\'e en la vida.
Mario, que entendi\'o todo.
Y en particular al Dr. Nahuel Andr\'es y el pelado que tiene en la pierna.
A Ari, Fede, Gaby, Ioni, Leo, Norbi, Jos\'e, ZZ, Exe, Juan, Nico y tantos m\'as, gracias por tantos momentos (l\'ease asados) compartidos.

A los compa\~neros del grupo de altas energ\'ias de la UBA, a Gustavo, Laura, Gino, Ro, Flor Jonatan y en particular a Gast\'on y Hern\'an con los que hoy todav\'ia tengo la suerte de compartir equipo.
Tambi\'en al grupo de altas energ\'ias de la USC, Washington, Ines, Guillermo y en especial a Daniel Garc\'ia, crack de cracks.  

A los profesores de las materias que curs\'e durante el doctorado, en particular a Sergio Sciutto, Gast\'on Gutierrez y Agust\'in Gravano, que cambiaron mi forma de ver las cosas en mas de una oportunidad.

Un agradecimiento especial a Marta Pedernera cuya dedicaci\'on y paciencia evit\'o que me ahogue burocracia varias veces.

Por \'ultimo a la Facultad de Ciencias Exactas y Naturales y al CONICET, que brindaron los medios necesarios para que pueda completar esta etapa de mi formaci\'on cient\'ifica.

\vspace{1cm}
A todos ustedes les digo \textbf{GRACIAS}.